%\VignetteIndexEntry{Introduction to plm}
\documentclass{article}
\usepackage[T1]{fontenc}
\usepackage{ucs}
\usepackage[utf8]{inputenc}

\title{Introduction to plm}

\author{Yves Croissant & Giovanni Milo}
\usepackage{/usr/share/R/share/texmf/Sweave}
\begin{document}

\maketitle


\section{Introduction}

The aim of package \texttt{plm} is to provide an easy way to estimate
panel models. Some panel models may be estimated with package \texttt{nlme}
(\textit{non--linear mixed effect models}), but not in an intuitive
way for an econometrician.
\texttt{plm} provides methods to read panel data, to estimate a wide
range of models and to make some tests. 
This library is loaded using :

\begin{Schunk}
\begin{Sinput}
> library(plm)
\end{Sinput}
\end{Schunk}

This document illustrates the features  of  \texttt{plm}, using
data available in  package \texttt{Ecdat}. 

\begin{Schunk}
\begin{Sinput}
> library(Ecdat)
\end{Sinput}
\end{Schunk}

These data are used in  \textsc{Baltagi} (2001).

\section{Reading data}

With \texttt{plm}, data are stored in an object of class
\texttt{pdata.frame},  which is a \texttt{data.frame} with additional
attributes describing the structure of the data set.
A \texttt{pdata.frame} may be created from an ordinary \texttt{data.frame}
using the \texttt{pdata.frame} function or from a text file using the
\texttt{pread.table} function.


\subsection{Reading the data from a data.frame}

We illustrate the use of the \texttt{pdata.frame} function with the
\texttt{Produc} data :


\begin{Schunk}
\begin{Sinput}
> data(Produc)
> pdata.frame(Produc, "state", "year", "pprod")
\end{Sinput}
\end{Schunk}

The  \texttt{pdata.frame} function has  4 arguments :

\begin{itemize}
\item the name of the  \texttt{data.frame},
\item \texttt{id} : the individual index,
\item \texttt{time} : the time index,
\item \texttt{name} : the name under which the \texttt{pdata.frame}
  will be stored.
\end{itemize}

Observations are assumed to be sorted by individuals first, and by
period. The third argument is optional, if \texttt{NULL} a new
variable called \texttt{time} is added. The fourth argument is also
optional, if \texttt{NULL} the \texttt{pdata.frame} is stored under
the same name as the \texttt{data.frame}.


\begin{Schunk}
\begin{Sinput}
> data(Hedonic)
> pdata.frame(Hedonic, "townid")
\end{Sinput}
\end{Schunk}

In case of a balanced panel, the \texttt{id} may be the number of
individuals. In this case, two new variables (called \texttt{id} and
\texttt{time}) are added.

\begin{Schunk}
\begin{Sinput}
> data(Wages)
> pdata.frame(Wages, 595)
\end{Sinput}
\end{Schunk}

A description of the data is obtained using the \texttt{summary}
method :

\begin{Schunk}
\begin{Sinput}
> summary(Hedonic)
\end{Sinput}
\begin{Soutput}
________________________________________________________________________________ 
___________________________________ Indexes ____________________________________
Individual index : townid
Time index       : time
________________________________________________________________________________ 
_______________________________ Panel Dimensions _______________________________
Unbalanced Panel
Number of Individuals        :  92
Number of Time Obserbations  :  from 1 to 30
Total Number of Observations :  506
________________________________________________________________________________ 
__________________________ Time/Individual Variation ___________________________
no time variation   :  zn indus rad tax ptratio 
________________________________________________________________________________ 
____________________________ Descriptive Statistics ____________________________
       mv              crim                zn             indus        chas    
 Min.   : 8.517   Min.   : 0.00632   Min.   :  0.00   Min.   : 0.46   no :471  
 1st Qu.: 9.742   1st Qu.: 0.08205   1st Qu.:  0.00   1st Qu.: 5.19   yes: 35  
 Median : 9.962   Median : 0.25651   Median :  0.00   Median : 9.69            
 Mean   : 9.942   Mean   : 3.61352   Mean   : 11.36   Mean   :11.14            
 3rd Qu.:10.127   3rd Qu.: 3.67708   3rd Qu.: 12.50   3rd Qu.:18.10            
 Max.   :10.820   Max.   :88.97620   Max.   :100.00   Max.   :27.74            
                                                                               
      nox              rm             age              dis        
 Min.   :14.82   Min.   :12.68   Min.   :  2.90   Min.   :0.1219  
 1st Qu.:20.16   1st Qu.:34.64   1st Qu.: 45.02   1st Qu.:0.7420  
 Median :28.94   Median :38.55   Median : 77.50   Median :1.1655  
 Mean   :32.11   Mean   :39.99   Mean   : 68.57   Mean   :1.1880  
 3rd Qu.:38.94   3rd Qu.:43.87   3rd Qu.: 94.07   3rd Qu.:1.6464  
 Max.   :75.86   Max.   :77.09   Max.   :100.00   Max.   :2.4954  
                                                                  
      rad             tax           ptratio          blacks       
 Min.   :0.000   Min.   :187.0   Min.   :12.60   Min.   :0.00032  
 1st Qu.:1.386   1st Qu.:279.0   1st Qu.:17.40   1st Qu.:0.37538  
 Median :1.609   Median :330.0   Median :19.05   Median :0.39144  
 Mean   :1.868   Mean   :408.2   Mean   :18.46   Mean   :0.35667  
 3rd Qu.:3.178   3rd Qu.:666.0   3rd Qu.:20.20   3rd Qu.:0.39623  
 Max.   :3.178   Max.   :711.0   Max.   :22.00   Max.   :0.39690  
                                                                  
     lstat             townid         time    
 Min.   :-4.0582   29     : 30   1      : 92  
 1st Qu.:-2.6659   84     : 23   2      : 75  
 Median :-2.1747   5      : 22   3      : 60  
 Mean   :-2.2342   83     : 19   4      : 50  
 3rd Qu.:-1.7744   41     : 18   5      : 39  
 Max.   :-0.9684   28     : 15   6      : 33  
                   (Other):379   (Other):157  
\end{Soutput}
\end{Schunk}

The printing consists on four sections :

\begin{itemize}
\item \texttt{indexes} indicates the names of the index variables,
\item \texttt{panel dimensions} gives information about the dimension
  of the panel,
\item \texttt{Time/individual variation} indicates whether some
  variables have only individual or time variation,
\item \texttt{Descriptive statistics} gives descriptive statistics
  about the variables.
\end{itemize}

\subsection{Reading the data from a text file}

\texttt{pread.table} reads panel data from a text file, with the
following syntax : 

\begin{verbatim}
pread.table("c:/mes documents/essai/mydata.txt",
         "firm","year","dataname",header=T,sep=";",dec=",")
\end{verbatim}

The arguments of  \texttt{pread.table} are :

\begin{itemize}
\item the text file,
\item \texttt{id} : the individual index,
\item \texttt{time} : the time index,
\item \texttt{name} : the name under which the  \texttt{pdata.frame}
  will be stored  (if \texttt{NULL}, the name of the \texttt{pdata.frame}
  is the name of the file without the path and the extension),
\item further arguments that will be passed to  \texttt{read.table}.
\end{itemize}




\section{Model estimation}

\texttt{plm} provides four functions for estimation :

\begin{itemize}
\item \texttt{plm} : estimation of the
  basic panel models, \emph{i.e.} within, between and random effect
  models. Models are estimated using the \texttt{lm} function to
  transformed data,
\item \texttt{pvcm} : estimation of models with variable coefficients,
\item \texttt{pgmm} : estimation of general method of moments models,
\item \texttt{pggls} : estimation of general feasible generalized least squares
  models.
\end{itemize}

All these functions share the same 4 first arguments :

\begin{itemize}
\item \texttt{formula} : the symbolic description of the model to be estimated,
\item \texttt{data} : the \texttt{pdata.frame} containing the data,
\item \texttt{effect} : the kind of effects to include in the model,
  \emph{i.e.} individual effects, time effects or both,
\item \texttt{model} : the kind of model to be estimated, most of the
time a model with fixed effects or a model with random effects.
\end{itemize}

The results of this four functions are stored in an object which class has the
same name of the function. They all inherit from class \texttt{panelmodel}. A
\texttt{panelmodel} object contains : \texttt{coefficients},
\texttt{residuals}, \texttt{fitted.values}, \texttt{vcov},
\texttt{df.residual} and \texttt{call}.

Functions that extract these elements and to print the object are provided.




\subsection{Estimation of the basic models with plm}

There are two ways to use \texttt{plm} : the first one is to estimate
a list of models (the default behavior), the second to estimate just one model.
In the first case, the estimated models are :

\begin{itemize}
\item the fixed effects model (\texttt{within}),
\item the pooling model (\texttt{pooling}),
\item the between model (\texttt{between}),
\item the error components model (\texttt{random}).
\end{itemize}

The basic use of \texttt{plm} is to indicate the model formula and the \texttt{pdata.frame}
\footnote{The following example is from \textsc{Baltagi} (2001), pp. 25--28.} :

\begin{Schunk}
\begin{Sinput}
> zz <- plm(log(gsp) ~ log(pcap) + log(pc) + log(emp) + unemp, 
+     data = pprod)
\end{Sinput}
\end{Schunk}

The result of the estimation is stored in a \texttt{plms} object which
is a list of 4 estimated models, each of them being objects of class \texttt{plm}.
Each individual model can be easily extracted :

\begin{Schunk}
\begin{Sinput}
> zzwith <- zz$within
\end{Sinput}
\end{Schunk}

A particular model to be estimated may also be indicated by filling
the \texttt{model} argument of \texttt{plm}.

\begin{Schunk}
\begin{Sinput}
> zzra <- plm(log(gsp) ~ log(pcap) + log(pc) + log(emp) + unemp, 
+     data = pprod, model = "random")
\end{Sinput}
\end{Schunk}


\begin{Schunk}
\begin{Sinput}
> print(zzra)
\end{Sinput}
\begin{Soutput}
Model Formula: log(gsp) ~ log(pcap) + log(pc) + log(emp) + unemp

Coefficients:
(intercept)   log(pcap)     log(pc)    log(emp)       unemp 
  2.1354110   0.0044386   0.3105484   0.7296705  -0.0061725 
\end{Soutput}
\end{Schunk}

\texttt{summary} and \texttt{print.summary} methods are provided. 

\begin{itemize}
\item for  \texttt{plms} objects, coefficients and standard errors
  of the fixed effects and the error components models are printed,
\item for  \texttt{plm} object, the table of coefficients and some
  statistics are printed.
\end{itemize}


\begin{Schunk}
\begin{Sinput}
> summary(zz)
\end{Sinput}
\begin{Soutput}
______________________________________________________________________ 
_________________________ Model Description __________________________
Oneway (individual) effect

Model Formula        : log(gsp) ~ log(pcap) + log(pc) + log(emp) + 
                           unemp
______________________________________________________________________ 
__________________________ Panel Dimensions __________________________
Balanced Panel
Number of Individuals        :  48
Number of Time Obserbations  :  17
Total Number of Observations :  816
______________________________________________________________________ 
____________________________ Coefficients ____________________________
                 within         wse      random    rse
(intercept)           .           .  2.13541100 0.1335
log(pcap)   -0.02614965  0.02813133  0.00443859 0.0234
log(pc)      0.29200693  0.02436591  0.31054843 0.0198
log(emp)     0.76815947  0.02918878  0.72967053 0.0249
unemp       -0.00529774  0.00095906 -0.00617247 0.0009
______________________________________________________________________ 
_______________________________ Tests ________________________________
Hausman Test                   : chi2(4) = 190.8961 (p.value=0)
F Test                         : F(47,764) = 75.8204 (p.value=0)
Lagrange Multiplier Test       : chi2(1) = 4134.961 (p.value=0)
______________________________________________________________________ 
\end{Soutput}
\begin{Sinput}
> summary(zzra)
\end{Sinput}
\begin{Soutput}
______________________________________________________________________ 
_________________________ Model Description __________________________
Oneway (individual) effect
Random Effect Model (Swamy-Arora's transformation)
Model Formula            : log(gsp) ~ log(pcap) + log(pc) + log(emp) + 
                               unemp
______________________________________________________________________ 
__________________________ Panel Dimensions __________________________
Balanced Panel
Number of Individuals        :  48
Number of Time Obserbations  :  17
Total Number of Observations :  816
______________________________________________________________________ 
______________________________ Effects _______________________________
                    var   std.dev  share
idiosyncratic 0.0014544 0.0381371 0.1754
individual    0.0068377 0.0826905 0.8246
theta   :  0.88884  
______________________________________________________________________ 
_____________________________ Residuals ______________________________
     Min.   1st Qu.    Median      Mean   3rd Qu.      Max. 
-1.07e-01 -2.46e-02 -2.37e-03 -9.93e-19  2.17e-02  2.00e-01 
______________________________________________________________________ 
____________________________ Coefficients ____________________________
               Estimate  Std. Error z-value  Pr(>|z|)    
(intercept)  2.13541100  0.13346149 16.0002 < 2.2e-16 ***
log(pcap)    0.00443859  0.02341732  0.1895    0.8497    
log(pc)      0.31054843  0.01980475 15.6805 < 2.2e-16 ***
log(emp)     0.72967053  0.02492022 29.2803 < 2.2e-16 ***
unemp       -0.00617247  0.00090728 -6.8033 1.023e-11 ***
---
Signif. codes:  0 ‘***’ 0.001 ‘**’ 0.01 ‘*’ 0.05 ‘.’ 0.1 ‘ ’ 1 
______________________________________________________________________ 
_________________________ Overall Statistics _________________________
Total Sum of Squares       : 29.209
Sum of Squares Residuals   : 1.1879
Rsq                        : 0.95933
F                          : 4782.77
P(F>0)                     : 8.76231e-08
______________________________________________________________________ 
\end{Soutput}
\end{Schunk}

For a \texttt{random} model, the \texttt{summary} method gives
information about the variance of the components of the errors.

\texttt{plm} objects can be updated using the \texttt{update} method :

\begin{Schunk}
\begin{Sinput}
> zzwithmod <- update(zzwith, . ~ . - unemp - log(emp) + emp)
> zzmod <- update(zz, . ~ . - unemp - log(emp) + emp)
> summary(zzwithmod)
\end{Sinput}
\begin{Soutput}
______________________________________________________________________ 
_________________________ Model Description __________________________
Oneway (individual) effect

Model Formula        : log(gsp) ~ log(pcap) + log(pc) + emp
______________________________________________________________________ 
__________________________ Panel Dimensions __________________________
Balanced Panel
Number of Individuals        :  48
Number of Time Obserbations  :  17
Total Number of Observations :  816
______________________________________________________________________ 
____________________________ Coefficients ____________________________
                within        wse     random       rse
(intercept)          .          . 7.1982e-01    0.1846
log(pcap)   1.7888e-01 3.9471e-02 3.4357e-01    0.0322
log(pc)     6.9975e-01 2.8280e-02 6.0369e-01    0.0256
emp         3.7909e-05 8.5192e-06 5.0924e-05 8.218e-06
______________________________________________________________________ 
_______________________________ Tests ________________________________
Hausman Test                   : chi2(3) = -32.23348 (p.value=1)
F Test                         : F(47,765) = 101.9109 (p.value=0)
Lagrange Multiplier Test       : chi2(1) = 4355.292 (p.value=0)
______________________________________________________________________ 
\end{Soutput}
\end{Schunk}

Fixed effects may be extracted easily from a \texttt{plms} or a
\texttt{plm} object using  \texttt{FE} :

\begin{Schunk}
\begin{Sinput}
> FE(zzmod)[1:10]
\end{Sinput}
\begin{Soutput}
    ALABAMA     ARIZONA    ARKANSAS  CALIFORNIA    COLORADO CONNECTICUT 
   1.171753    1.306239    1.187700    1.619198    1.458215    1.706034 
   DELAWARE     FLORIDA     GEORGIA       IDAHO 
   1.203575    1.556497    1.446017    1.100205 
\end{Soutput}
\end{Schunk}

The \texttt{FE} function returns an object of class \texttt{FE}. A
summary method is provided, which prints the effects (in deviation
from the overall intercept), their standard
errors and the test of equality to the overall intercept.

\begin{Schunk}
\begin{Sinput}
> summary(FE(zzmod))[1:10, ]
\end{Sinput}
\begin{Soutput}
                     FE std.error     t-value    p-value
ALABAMA     -0.15044698 0.2142832 -0.70209405 0.48262051
ARIZONA     -0.01596112 0.2115486 -0.07544893 0.93985753
ARKANSAS    -0.13449962 0.2009406 -0.66935022 0.50327210
CALIFORNIA   0.29699815 0.2450846  1.21181889 0.22558172
COLORADO     0.13601482 0.2109386  0.64480772 0.51905180
CONNECTICUT  0.38383408 0.2155489  1.78072876 0.07495677
DELAWARE    -0.11862549 0.1892258 -0.62689921 0.53072531
FLORIDA      0.23429687 0.2269427  1.03240541 0.30188224
GEORGIA      0.12381708 0.2193786  0.56439904 0.57248259
IDAHO       -0.22199517 0.1852999 -1.19803151 0.23090475
\end{Soutput}
\end{Schunk}


\subsection{More advanced use of plm}


\subsubsection{Options for the random effect model}

The random effect model is obtained as a linear estimation on
quasi--differentiated  data. The parameter of this transformation is
obtained using preliminary estimations. Four estimators of this
parameter are available, depending on the value of the argument \texttt{random.method}  :

\begin{itemize}
\item \texttt{swar} : from \textsc{Swamy} and \textsc{Arora}
  (1972), the default value,
\item \texttt{walhus} : from \textsc{Wallace} and \textsc{Hussain} (1969),
\item \texttt{amemiya} : from \textsc{Amemiyia} (1971),
\item \texttt{nerlove} : from \textsc{Nerlove} (1971).
\end{itemize}

For exemple, to use the \texttt{amemiya} estimator :

\begin{Schunk}
\begin{Sinput}
> zzra <- plm(log(gsp) ~ log(pcap) + log(pc) + log(emp) + unemp, 
+     data = pprod, model = "random", random.method = "amemiya")
\end{Sinput}
\end{Schunk}


\subsubsection{Choosing  the effects}

The default behavior of \texttt{plm} is to introduce individual
effects. Using the \texttt{effect} argument, one may also introduce :

\begin{itemize}
\item time effects (\texttt{effect="time"}),
\item individual and time effects (\texttt{effect="twoways"}).
\end{itemize}

For example, to estimate a two--ways effect model for the
\texttt{Grunfeld} data :

\begin{Schunk}
\begin{Sinput}
> data(Grunfeld)
> pdata.frame(Grunfeld, "firm", "year")
> z <- plm(inv ~ value + capital, data = Grunfeld, effect = "twoways", 
+     random.method = "amemiya")
> summary(z$random)
\end{Sinput}
\begin{Soutput}
______________________________________________________________________ 
_________________________ Model Description __________________________
Twoways effects
Random Effect Model (Amemiya's transformation)
Model Formula            : inv ~ value + capital
______________________________________________________________________ 
__________________________ Panel Dimensions __________________________
Balanced Panel
Number of Individuals        :  10
Number of Time Obserbations  :  20
Total Number of Observations :  200
______________________________________________________________________ 
______________________________ Effects _______________________________
                   var  std.dev  share
idiosyncratic 2644.135   51.421 0.2359
individual    8294.716   91.075 0.7400
time           270.529   16.448 0.0241
theta  : 0.87475 (id) 0.29695 (time) 0.29595 (total)
______________________________________________________________________ 
_____________________________ Residuals ______________________________
     Min.   1st Qu.    Median      Mean   3rd Qu.      Max. 
-1.76e+02 -1.80e+01  3.02e+00 -3.56e-16  1.80e+01  2.33e+02 
______________________________________________________________________ 
____________________________ Coefficients ____________________________
              Estimate Std. Error z-value Pr(>|z|)    
(intercept) -64.351811  31.183651 -2.0636  0.03905 *  
value         0.111593   0.011028 10.1192  < 2e-16 ***
capital       0.324625   0.018850 17.2214  < 2e-16 ***
---
Signif. codes:  0 ‘***’ 0.001 ‘**’ 0.01 ‘*’ 0.05 ‘.’ 0.1 ‘ ’ 1 
______________________________________________________________________ 
_________________________ Overall Statistics _________________________
Total Sum of Squares       : 2038000
Sum of Squares Residuals   : 514120
Rsq                        : 0.74774
F                          : 291.965
P(F>0)                     : 0.00341914
______________________________________________________________________ 
\end{Soutput}
\end{Schunk}

In the ``effects'' section of the result is printed now the variance
of the three elements of the error term and the three parameters used
in the transformation. 

The two--ways effect model is for the moment only available for
balanced panels.


\subsubsection{Hausman--Taylor's model}

\textsc{Hausman}--\textsc{Taylor}'s model may be estimated with \texttt{plm}
by equating the \texttt{model} argument to  \texttt{"ht"} and
filling the second argument \texttt{instruments} with a formula
indicating the variables used as instruments.


\begin{Schunk}
\begin{Sinput}
> data(Wages)
> pdata.frame(Wages, 595)
> form = lwage ~ wks + south + smsa + married + exp + I(exp^2) + 
+     bluecol + ind + union + sex + black + ed
> ht = plm(form, data = Wages, instruments = ~sex + black + bluecol + 
+     south + smsa + ind, model = "ht")
> summary(ht)
\end{Sinput}
\begin{Soutput}
______________________________________________________________________ 
_________________________ Model Description __________________________
Oneway (individual) effect
Hausman-Taylor Model
Model Formula            : lwage ~ wks + south + smsa + married + 
                               exp + I(exp^2) + bluecol + ind + 
                               union + sex + black + ed
Instrumental Variables   : ~sex + black + bluecol + south + smsa + 
                               ind
Time--Varying Variables    
    exogenous variables   :  bluecolyes,southyes,smsayes,ind 
    endogenous variables  :  wks,marriedyes,exp,I(exp^2),unionyes 
Time--Invariant Variables  
    exogenous variables   :  sexmale,blackyes 
    endogenous variables  :  ed 
______________________________________________________________________ 
__________________________ Panel Dimensions __________________________
Balanced Panel
Number of Individuals        :  595
Number of Time Obserbations  :  7
Total Number of Observations :  4165
______________________________________________________________________ 
______________________________ Effects _______________________________
                   var  std.dev  share
idiosyncratic 0.023044 0.151803 0.0253
individual    0.886993 0.941803 0.9747
theta   :  0.93919  
______________________________________________________________________ 
_____________________________ Residuals ______________________________
     Min.   1st Qu.    Median      Mean   3rd Qu.      Max. 
-1.92e+00 -7.07e-02  6.57e-03 -2.46e-17  7.97e-02  2.03e+00 
______________________________________________________________________ 
____________________________ Coefficients ____________________________
               Estimate  Std. Error z-value  Pr(>|z|)    
(intercept)  2.7818e+00  3.0765e-01  9.0422 < 2.2e-16 ***
wks          8.3740e-04  5.9973e-04  1.3963   0.16263    
southyes     7.4398e-03  3.1955e-02  0.2328   0.81590    
smsayes     -4.1833e-02  1.8958e-02 -2.2066   0.02734 *  
marriedyes  -2.9851e-02  1.8980e-02 -1.5728   0.11578    
exp          1.1313e-01  2.4710e-03 45.7851 < 2.2e-16 ***
I(exp^2)    -4.1886e-04  5.4598e-05 -7.6718 1.688e-14 ***
bluecolyes  -2.0705e-02  1.3781e-02 -1.5024   0.13299    
ind          1.3604e-02  1.5237e-02  0.8928   0.37196    
unionyes     3.2771e-02  1.4908e-02  2.1982   0.02794 *  
sexmale      1.3092e-01  1.2666e-01  1.0337   0.30129    
blackyes    -2.8575e-01  1.5570e-01 -1.8352   0.06647 .  
ed           1.3794e-01  2.1248e-02  6.4919 8.474e-11 ***
---
Signif. codes:  0 ‘***’ 0.001 ‘**’ 0.01 ‘*’ 0.05 ‘.’ 0.1 ‘ ’ 1 
______________________________________________________________________ 
_________________________ Overall Statistics _________________________
Total Sum of Squares       : 243.04
Sum of Squares Residuals   : 95.947
Rsq                        : 0.60522
F                          : 489.524
P(F>0)                     : 3.33067e-16
______________________________________________________________________ 
\end{Soutput}
\end{Schunk}

\subsubsection{Instrumental variables estimation}

One or all of the models may be estimated using instrumental variables
by indicating the list of the instrumental variables. This can be done
using one of the two following techniques :

\begin{itemize}
\item specifying the total list of instruments  (using the
  \texttt{instruments} argument of \texttt{plm}),
\item specifying, on the one hand the external instruments in the argument
  \texttt{instrument} and on  the other hand the variables of the
  model that are assumed to be endogenous in the argument \texttt{endog}.
\end{itemize}

The instrumental variables estimator used may be indicated with the
\texttt{inst.method} argument :
\begin{itemize}
\item \texttt{bvk}, from  \textsc{Balestra} et
  \textsc{Varadharajan--Krishnakumar} (1987), the default value,
\item \texttt{baltagi}, from \textsc{Baltagi} (1981).
\end{itemize}

We illustrate instrumental variables estimation with the
\texttt{Crime} data\footnote{See
  \textsc{Baltagi} (2001), pp.119--120.}. 
The same estimation is done using the first syntax  (\texttt{cr1}) and
the second (\texttt{cr2}). The  \texttt{prbarr} and \texttt{polpc}
variables are
assumed to be endogenous and there are two external instruments \texttt{taxpc} and \texttt{mix} :

\begin{Schunk}
\begin{Sinput}
> data(Crime)
> pdata.frame(Crime, "county", "year")
> form = log(crmrte) ~ log(prbarr) + log(polpc) + log(prbconv) + 
+     log(prbpris) + log(avgsen) + log(density) + log(wcon) + log(wtuc) + 
+     log(wtrd) + log(wfir) + log(wser) + log(wmfg) + log(wfed) + 
+     log(wsta) + log(wloc) + log(pctymle) + log(pctmin) + region + 
+     smsa + year
> inst = ~log(prbconv) + log(prbpris) + log(avgsen) + log(density) + 
+     log(wcon) + log(wtuc) + log(wtrd) + log(wfir) + log(wser) + 
+     log(wmfg) + log(wfed) + log(wsta) + log(wloc) + log(pctymle) + 
+     log(pctmin) + region + smsa + log(taxpc) + log(mix) + year
> inst2 = ~log(taxpc) + log(mix)
> endog = ~log(prbarr) + log(polpc)
> cr = plm(form, data = Crime)
> cr1 = plm(form, data = Crime, instruments = inst)
> cr2 = plm(form, data = Crime, instruments = inst2, endog = endog)
> summary(cr2$random)
\end{Sinput}
\begin{Soutput}
______________________________________________________________________ 
_________________________ Model Description __________________________
Oneway (individual) effect
Random Effect Model (Swamy-Arora's transformation)
Instrumental variable estimation (Balestra-Varadharajan-Krishnakumar's transformation)
Model Formula            : log(crmrte) ~ log(prbarr) + log(polpc) + 
                               log(prbconv) + log(prbpris) + 
                               log(avgsen) + log(density) + log(wcon) + 
                               log(wtuc) + log(wtrd) + log(wfir) + 
                               log(wser) + log(wmfg) + log(wfed) + 
                               log(wsta) + log(wloc) + log(pctymle) + 
                               log(pctmin) + region + smsa + 
                               year
Endogenous Variables     : ~log(prbarr) + log(polpc)
Instrumental Variables   : ~log(taxpc) + log(mix)
______________________________________________________________________ 
__________________________ Panel Dimensions __________________________
Balanced Panel
Number of Individuals        :  90
Number of Time Obserbations  :  7
Total Number of Observations :  630
______________________________________________________________________ 
______________________________ Effects _______________________________
                   var  std.dev share
idiosyncratic 0.022269 0.149228 0.326
individual    0.046036 0.214561 0.674
theta   :  0.74576  
______________________________________________________________________ 
_____________________________ Residuals ______________________________
     Min.   1st Qu.    Median      Mean   3rd Qu.      Max. 
-5.02e+00 -4.76e-01  2.73e-02  7.11e-16  5.26e-01  3.19e+00 
______________________________________________________________________ 
____________________________ Coefficients ____________________________
                Estimate Std. Error z-value  Pr(>|z|)    
(intercept)   -0.4538241  1.7029840 -0.2665  0.789864    
log(prbarr)   -0.4141200  0.2210540 -1.8734  0.061015 .  
log(polpc)     0.5049285  0.2277811  2.2167  0.026642 *  
log(prbconv)  -0.3432383  0.1324679 -2.5911  0.009567 ** 
log(prbpris)  -0.1900437  0.0733420 -2.5912  0.009564 ** 
log(avgsen)   -0.0064374  0.0289406 -0.2224  0.823977    
log(density)   0.4343519  0.0711528  6.1045 1.031e-09 ***
log(wcon)     -0.0042963  0.0414225 -0.1037  0.917392    
log(wtuc)      0.0444572  0.0215449  2.0635  0.039068 *  
log(wtrd)     -0.0085626  0.0419822 -0.2040  0.838387    
log(wfir)     -0.0040302  0.0294565 -0.1368  0.891175    
log(wser)      0.0105604  0.0215822  0.4893  0.624620    
log(wmfg)     -0.2017917  0.0839423 -2.4039  0.016220 *  
log(wfed)     -0.2134634  0.2151074 -0.9924  0.321023    
log(wsta)     -0.0601083  0.1203146 -0.4996  0.617362    
log(wloc)      0.1835137  0.1396721  1.3139  0.188884    
log(pctymle)  -0.1458448  0.2268137 -0.6430  0.520214    
log(pctmin)    0.1948760  0.0459409  4.2419 2.217e-05 ***
regionwest    -0.2281780  0.1010317 -2.2585  0.023916 *  
regioncentral -0.1987675  0.0607510 -3.2718  0.001068 ** 
smsayes       -0.2595423  0.1499780 -1.7305  0.083535 .  
year82         0.0132140  0.0299923  0.4406  0.659518    
year83        -0.0847676  0.0320008 -2.6489  0.008075 ** 
year84        -0.1062004  0.0387893 -2.7379  0.006184 ** 
year85        -0.0977398  0.0511685 -1.9102  0.056113 .  
year86        -0.0719390  0.0605821 -1.1875  0.235045    
year87        -0.0396520  0.0758537 -0.5227  0.601153    
---
Signif. codes:  0 ‘***’ 0.001 ‘**’ 0.01 ‘*’ 0.05 ‘.’ 0.1 ‘ ’ 1 
______________________________________________________________________ 
_________________________ Overall Statistics _________________________
Total Sum of Squares       : 1354.7
Sum of Squares Residuals   : 557.64
Rsq                        : 0.58836
F                          : 33.1494
P(F>0)                     : 7.77156e-16
______________________________________________________________________ 
\end{Soutput}
\end{Schunk}



\subsubsection{Unbalanced panel}

\texttt{plm} enables the estimation of unbalanced panel data, with a
few restrictions (twoways effects models are not supported and the
only transformation for random effects models is \texttt{swar}).

The
following example is based on the \texttt{Hedonic} data\footnote{See \textsc{Baltagi}
  (2001), p. 174.}:

\begin{Schunk}
\begin{Sinput}
> form = mv ~ crim + zn + indus + chas + nox + rm + age + dis + 
+     rad + tax + ptratio + blacks + lstat
> ba = plm(form, data = Hedonic)
> summary(ba$random)
\end{Sinput}
\begin{Soutput}
______________________________________________________________________ 
_________________________ Model Description __________________________
Oneway (individual) effect
Random Effect Model (Swamy-Arora's transformation)
Model Formula            : mv ~ crim + zn + indus + chas + nox + 
                               rm + age + dis + rad + tax + ptratio + 
                               blacks + lstat
______________________________________________________________________ 
__________________________ Panel Dimensions __________________________
Unbalanced Panel
Number of Individuals        :  92
Number of Time Obserbations  :  from 1 to 30
Total Number of Observations :  506
______________________________________________________________________ 
______________________________ Effects _______________________________
                   var  std.dev share
idiosyncratic 0.016965 0.130249 0.502
individual    0.016832 0.129738 0.498
theta  : 
   Min. 1st Qu.  Median    Mean 3rd Qu.    Max. 
 0.2915  0.5904  0.6655  0.6499  0.7447  0.8197 
______________________________________________________________________ 
_____________________________ Residuals ______________________________
     Min.   1st Qu.    Median      Mean   3rd Qu.      Max. 
-0.641000 -0.066100 -0.000519 -0.001990  0.069800  0.527000 
______________________________________________________________________ 
____________________________ Coefficients ____________________________
               Estimate  Std. Error  z-value  Pr(>|z|)    
(intercept)  9.6778e+00  2.0714e-01  46.7207 < 2.2e-16 ***
crim        -7.2338e-03  1.0346e-03  -6.9921 2.707e-12 ***
zn           3.9575e-05  6.8778e-04   0.0575 0.9541153    
indus        2.0794e-03  4.3403e-03   0.4791 0.6318706    
chasyes     -1.0591e-02  2.8960e-02  -0.3657 0.7145720    
nox         -5.8630e-03  1.2455e-03  -4.7074 2.509e-06 ***
rm           9.1773e-03  1.1792e-03   7.7828 7.105e-15 ***
age         -9.2715e-04  4.6468e-04  -1.9952 0.0460159 *  
dis         -1.3288e-01  4.5683e-02  -2.9088 0.0036279 ** 
rad          9.6863e-02  2.8350e-02   3.4168 0.0006337 ***
tax         -3.7472e-04  1.8902e-04  -1.9824 0.0474298 *  
ptratio     -2.9723e-02  9.7538e-03  -3.0473 0.0023089 ** 
blacks       5.7506e-01  1.0103e-01   5.6920 1.256e-08 ***
lstat       -2.8514e-01  2.3855e-02 -11.9533 < 2.2e-16 ***
---
Signif. codes:  0 ‘***’ 0.001 ‘**’ 0.01 ‘*’ 0.05 ‘.’ 0.1 ‘ ’ 1 
______________________________________________________________________ 
_________________________ Overall Statistics _________________________
Total Sum of Squares       : 893.08
Sum of Squares Residuals   : 8.6843
Rsq                        : 0.99028
F                          : 3854.18
P(F>0)                     : 0
______________________________________________________________________ 
\end{Soutput}
\end{Schunk}


\subsection{Variable coefficients model}

The \texttt{pvcm} function enables the estimation of variable
cofficients models. Time or individual effects are introduced if
\texttt{effect} is fixed to \texttt{"time"} or \texttt{"individual"}
(the default value). 

Coefficients are assumed to be fixed if \texttt{model="within"} and
random if \texttt{model="random"}. In the first case, a different
model is estimated for each individual (or time period). In the second
case, the \textsc{Swamy} (1970) model is estimated. It is a
generalized least squares model which use the result of the previous model.


With the \texttt{Grunfeld} data, we get :

\begin{Schunk}
\begin{Sinput}
> znp <- pvcm(inv ~ value + capital, data = Grunfeld, model = "within")
> znp
\end{Sinput}
\begin{Soutput}
Model Formula: inv ~ value + capital

Coefficients:
   (Intercept)     value   capital
1   -149.78245 0.1192808 0.3714448
2    -49.19832 0.1748560 0.3896419
3     -9.95631 0.0265512 0.1516939
4     -6.18996 0.0779478 0.3157182
5     22.70712 0.1623777 0.0031017
6     -8.68554 0.1314548 0.0853743
7     -4.49953 0.0875272 0.1237814
8     -0.50939 0.0528941 0.0924065
9     -7.72284 0.0753879 0.0821036
10     0.16152 0.0045734 0.4373692
\end{Soutput}
\begin{Sinput}
> summary(znp)
\end{Sinput}
\begin{Soutput}
______________________________________________________________________ 
_________________________ Model Description __________________________
Oneway (individual) effect
No-pooling model
Model Formula             : inv ~ value + capital
______________________________________________________________________ 
__________________________ Panel Dimensions __________________________
Balanced Panel
Number of Individuals        :  10
Number of Time Obserbations  :  20
Total Number of Observations :  200
______________________________________________________________________ 
_____________________________ Residuals ______________________________
     Min.   1st Qu.    Median      Mean   3rd Qu.      Max. 
-1.84e+02 -7.12e+00 -3.93e-01  3.44e-16  5.70e+00  1.44e+02 
______________________________________________________________________ 
____________________________ Coefficients ____________________________
  (Intercept)          value            capital      
 Min.   :-149.78   Min.   :0.00457   Min.   :0.0031  
 1st Qu.:  -9.64   1st Qu.:0.05852   1st Qu.:0.0871  
 Median :  -6.96   Median :0.08274   Median :0.1377  
 Mean   : -21.37   Mean   :0.09129   Mean   :0.2053  
 3rd Qu.:  -1.51   3rd Qu.:0.12841   3rd Qu.:0.3575  
 Max.   :  22.71   Max.   :0.17486   Max.   :0.4374  
______________________________________________________________________ 
_________________________ Overall Statistics _________________________
Total Sum of Squares       : 9359900
Sum of Squares Residuals   : 324730
Rsq                        : 0.96531
______________________________________________________________________ 
\end{Soutput}
\end{Schunk}

\begin{Schunk}
\begin{Sinput}
> form <- inv ~ value + capital
> sw <- plm(form, data = Grunfeld, model = "random")
> summary(sw)
\end{Sinput}
\begin{Soutput}
______________________________________________________________________ 
_________________________ Model Description __________________________
Oneway (individual) effect
Random Effect Model (Swamy-Arora's transformation)
Model Formula            : inv ~ value + capital
______________________________________________________________________ 
__________________________ Panel Dimensions __________________________
Balanced Panel
Number of Individuals        :  10
Number of Time Obserbations  :  20
Total Number of Observations :  200
______________________________________________________________________ 
______________________________ Effects _______________________________
                   var  std.dev share
idiosyncratic 2784.458   52.768 0.282
individual    7089.800   84.201 0.718
theta   :  0.86122  
______________________________________________________________________ 
_____________________________ Residuals ______________________________
     Min.   1st Qu.    Median      Mean   3rd Qu.      Max. 
-1.78e+02 -1.97e+01  4.69e+00  3.92e-16  1.95e+01  2.53e+02 
______________________________________________________________________ 
____________________________ Coefficients ____________________________
              Estimate Std. Error z-value Pr(>|z|)    
(intercept) -57.834415  28.898935 -2.0013  0.04536 *  
value         0.109781   0.010493 10.4627  < 2e-16 ***
capital       0.308113   0.017180 17.9339  < 2e-16 ***
---
Signif. codes:  0 ‘***’ 0.001 ‘**’ 0.01 ‘*’ 0.05 ‘.’ 0.1 ‘ ’ 1 
______________________________________________________________________ 
_________________________ Overall Statistics _________________________
Total Sum of Squares       : 2381400
Sum of Squares Residuals   : 548900
Rsq                        : 0.7695
F                          : 328.837
P(F>0)                     : 0.00303635
______________________________________________________________________ 
\end{Soutput}
\end{Schunk}

\subsection{General method of moments estimator}

The general method of moments is provided by the \texttt{pgmm}
function. It's main argument is either a formula or a variable,
entered as a character string. In this last case, a pure
autoregressive model is estimated. Lag values of the dependent
variable should not be written in the formula, but specified in the
\texttt{lags.endog} argument (as an integer 0 : no lag, 1 or 2).
The effect argument is either \texttt{NULL} (the default),
\texttt{"individual"} or \texttt{"twoways"}. In the first case, the
model is estimated in levels. In the second case, the model is
estimated in first differences to get rid of the individuals
effects. In the last case, the model is estimated in first differences
and time dummies are included. 

By default, the instruments are the
independent variables (and the dependent variables if \texttt{lags.endog}>0).

The complete list of instruments can also be specified with the
argument \texttt{instruments} which should be a one side formula. 
Note that if the \texttt{formula} contains lags of some variables, you
\emph{have} to write explicitely the list of instruments.

For each of the instruments, the number of lags (or/and leads) is
specified with the \texttt{first.period} and \texttt{last.period}
arguments. For example, if \texttt{first.period=-4} and
\texttt{last.period=-1}, $x_{t-4},x_{t-3}, x_{t-2}, x_{t-1}$ are used
as instruments for observation $t$. Use ``big'' integers (like -99 and
99) to include all lags and/or leads. The \texttt{first.period} and
\texttt{last.period} can be of length 1, 2 or J (the number of
instruments). In the first case, the same periods are chosen for all
the instruments. In the second case, two different values may be
chosen for the dependent variable and all the independent variables
(only relevant when \texttt{lags.endog}>0). In the last case,
different values may be chosen for each instrument. Instruments are
introduced in levels if \texttt{inst.transformation="l"} and in first
difference if \texttt{inst.transformation="d"}. Like previously, it can be
specified with a vector of length 1, 2 or J.

The \texttt{model} argument specifies whether a one--step or a
two--steps model is required (\texttt{"onestep"} or \texttt{"twosteps"}).

The  following example is from \textsc{Arellano} (2003). Employment in
different firms is explained by past values of employment and wages
(two lags). All available lags are used up to $t-2$.

\begin{Schunk}
\begin{Sinput}
> data(Snmesp)
> pdata.frame(Snmesp, "firm", "year")
> z <- pgmm(n ~ lag(w) + lag(w, 2), effect = "twoways", model = "twosteps", 
+     Snmesp, lags.endog = 2, last.period = c(-2), inst.transformation = c("l"), 
+     instruments = ~n + w)
> summary(z)
\end{Sinput}
\begin{Soutput}
______________________________________________________________________ 
_________________________ Model Description __________________________


Model Formula             : n ~ lag(w) + lag(w, 2)
______________________________________________________________________ 
__________________________ Panel Dimensions __________________________
Balanced Panel
Number of Individuals        :  738
Number of Time Obserbations  :  8
Total Number of Observations :  5904
______________________________________________________________________ 
_____________________________ Residuals ______________________________
     Min.   1st Qu.    Median      Mean   3rd Qu.      Max. 
-1.540000 -0.051100  0.001020  0.000175  0.055000  1.280000 
______________________________________________________________________ 
_________________________ Model Description __________________________
            Estimate Std. Error  z-value Pr(>|z|)
lag(n)     0.8415278   0.088389  9.52068 0.000000
lag(n,2)  -0.0031454   0.029044 -0.10829 0.913762
lag(w)     0.0779827   0.083638  0.93238 0.351141
lag(w, 2) -0.0525764   0.024942 -2.10796 0.035034
______________________________________________________________________ 
________________________ Specification tests _________________________
Sargan Test                   : chi2(36) = 36.914 (p.value=0.42648)
Wald test for time dummies    : chi2(5) = 44.476 (p.value=1.8536e-08)
\end{Soutput}
\end{Schunk}

In the following example, a pure auto--regressive model is
estimated. In this cas, the first argument is the name of the
variable, entered as a character string.

\begin{Schunk}
\begin{Sinput}
> z <- pgmm("n", effect = "twoways", model = "twosteps", Snmesp, 
+     lags.endog = 2, last.period = c(-2), inst.transformation = c("l"))
> summary(z)
\end{Sinput}
\begin{Soutput}
______________________________________________________________________ 
_________________________ Model Description __________________________


Model Formula             : "n"
______________________________________________________________________ 
__________________________ Panel Dimensions __________________________
Balanced Panel
Number of Individuals        :  738
Number of Time Obserbations  :  8
Total Number of Observations :  5904
______________________________________________________________________ 
_____________________________ Residuals ______________________________
     Min.   1st Qu.    Median      Mean   3rd Qu.      Max. 
-1.45e+00 -4.99e-02 -2.42e-04  6.63e-05  5.21e-02  1.20e+00 
______________________________________________________________________ 
_________________________ Model Description __________________________
         Estimate Std. Error z-value Pr(>|z|)
lag(n)   0.747547   0.088270  8.4688 0.000000
lag(n,2) 0.037680   0.021952  1.7165 0.086068
______________________________________________________________________ 
________________________ Specification tests _________________________
Sargan Test                   : chi2(18) = 14.409 (p.value=0.70206)
Wald test for time dummies    : chi2(5) = 59.156 (p.value=1.8156e-11)
\end{Soutput}
\end{Schunk}

\subsection{General FGLS models}
General FGLS estimators are based on a two-step estimation process: first an OLS model is estimated, then its residuals are used to estimate an error covariance matrix for use in a feasible-GLS analysis. Formally, the structure of the error covariance matrix is $ V=I_N \otimes \Omega $, with symmetry being the only requisite for $\Omega$: $ \Omega(ij)=\Omega(ji) $ (see Wooldridge (2002), 10.4.3 and 10.5.5).

This framework allows the error covariance structure inside every group (if \texttt{effect="individual"}) of observations to be fully unrestricted and is therefore robust against any type of intragroup heteroskedasticity and serial correlation. This structure, by converse, is assumed identical across groups and thus \texttt{ggls} is inefficient under groupwise heteroskedasticity. Cross-sectional correlation is excluded a priori.

Moreover, the number of variance parameters to be estimated with $NT$ data points is $T(T+1)/2$, which makes these estimators particularly suited for situations where $N>>T$, as e.g. in labour or household income surveys, while problematic for "long" panels. 

In a pooled time series context (\texttt{effect="time"}), symmetrically, this estimator is able to account for arbitrary cross-sectional correlation, provided that the latter is time-invariant (see Greene (2003) 13.9.1-2, p.321-2). In this case serial correlation has to be assumed away and the estimator is consistent with respect to the time dimension, keeping N fixed.

The function \texttt{pggls} estimates general FGLS models, with either fixed of ''random'' effects\footnote{The ''random effect'' is better termed ''general FGLS'' model, as in fact it does not have a proper random effects structure, but we keep this terminology for consistency with \texttt{plm}.}. 

The ''random effect'' general FGLS is estimated by
 
\begin{Schunk}
\begin{Sinput}
> zz <- pggls(log(gsp) ~ log(pcap) + log(pc) + log(emp) + unemp, 
+     data = pprod, model = "random")
> summary(zz)
\end{Sinput}
\begin{Soutput}
______________________________________________________________________ 
_________________________ Model Description __________________________
Oneway (individual) effect
Random effects model
Model Formula             : log(gsp) ~ log(pcap) + log(pc) + 
                                log(emp) + unemp
______________________________________________________________________ 
__________________________ Panel Dimensions __________________________
Balanced Panel
Number of Individuals        :  48
Number of Time Obserbations  :  17
Total Number of Observations :  816
______________________________________________________________________ 
_____________________________ Residuals ______________________________
    Min.  1st Qu.   Median     Mean  3rd Qu.     Max. 
-0.25600 -0.07020 -0.01410 -0.00891  0.03910  0.45500 
______________________________________________________________________ 
____________________________ Coefficients ____________________________
               Estimate  Std. Error z-value  Pr(>|z|)    
(intercept)  2.26388494  0.10077679 22.4643 < 2.2e-16 ***
log(pcap)    0.10566584  0.02004106  5.2725 1.346e-07 ***
log(pc)      0.21643137  0.01539471 14.0588 < 2.2e-16 ***
log(emp)     0.71293894  0.01863632 38.2553 < 2.2e-16 ***
unemp       -0.00447265  0.00045214 -9.8921 < 2.2e-16 ***
---
Signif. codes:  0 ‘***’ 0.001 ‘**’ 0.01 ‘*’ 0.05 ‘.’ 0.1 ‘ ’ 1 
______________________________________________________________________ 
_________________________ Overall Statistics _________________________
Total Sum of Squares       : 849.81
Sum of Squares Residuals   : 7.5587
Rsq                        : 0.99111
______________________________________________________________________ 
\end{Soutput}
\end{Schunk}

The fixed effects \texttt{pggls} (see \textsc{Wooldridge} (2002, p.276)) is based on estimation of a within model in the first step; the rest follows as above. It is estimated by

\begin{Schunk}
\begin{Sinput}
> zz <- pggls(log(gsp) ~ log(pcap) + log(pc) + log(emp) + unemp, 
+     data = pprod, model = "within")
> summary(zz)
\end{Sinput}
\begin{Soutput}
______________________________________________________________________ 
_________________________ Model Description __________________________
Oneway (individual) effect
Within model
Model Formula             : log(gsp) ~ log(pcap) + log(pc) + 
                                log(emp) + unemp
______________________________________________________________________ 
__________________________ Panel Dimensions __________________________
Balanced Panel
Number of Individuals        :  48
Number of Time Obserbations  :  17
Total Number of Observations :  816
______________________________________________________________________ 
_____________________________ Residuals ______________________________
     Min.   1st Qu.    Median      Mean   3rd Qu.      Max. 
-1.18e-01 -2.37e-02 -4.72e-03  2.92e-17  1.73e-02  1.78e-01 
______________________________________________________________________ 
____________________________ Coefficients ____________________________
             Estimate  Std. Error z-value  Pr(>|z|)    
log(pcap) -0.00104277  0.02900641 -0.0359    0.9713    
log(pc)    0.17151298  0.01807934  9.4867 < 2.2e-16 ***
log(emp)   0.84449144  0.02042362 41.3488 < 2.2e-16 ***
unemp     -0.00357102  0.00047319 -7.5468 4.463e-14 ***
---
Signif. codes:  0 ‘***’ 0.001 ‘**’ 0.01 ‘*’ 0.05 ‘.’ 0.1 ‘ ’ 1 
______________________________________________________________________ 
_________________________ Overall Statistics _________________________
Total Sum of Squares       : 18.941
Sum of Squares Residuals   : 1.1623
Rsq                        : 0.93864
______________________________________________________________________ 
\end{Soutput}
\end{Schunk}

The \texttt{pggls} function is similar to \texttt{plm} in many respects (e.g., Hausman tests may be carried out on \texttt{pggls} objects much the same way they are done on \texttt{plm} ones). An exception is that the estimate of the group covariance matrix of errors (\verb!zz$sigma!, 17x17 matrix, not shown) is reported in the model objects instead of the usual estimated variances of the two error components.

\section{Tests}


\subsection{Tests of poolability}

\texttt{pooltest} tests the hypothesis that the same coefficients
apply to each individual. It is a standard F test, based on the
comparison of a model obtained for the full sample and a model based
on the estimation of an equation for each individual. The main
argument of \texttt{pooltest} is a \texttt{plms} or a \texttt{plm} object. 
The second argument is a \texttt{pvcm} object obtained with \texttt{model=within} .
If the first argument is a \texttt{plms} object, 
a third argument  \texttt{effect} should be fixed to \texttt{FALSE} if
the intercepts are assumed to be identical (the default value) or \texttt{TRUE} if
not\footnote{The following examples are from 
  \textsc{Baltagi} (2001), pp. 57--58.}.

\begin{Schunk}
\begin{Sinput}
> form = inv ~ value + capital
> znp = pvcm(form, data = Grunfeld, model = "within")
> zplm = plm(form, data = Grunfeld)
> pooltest(zplm, znp)
\end{Sinput}
\begin{Soutput}
	F statistic

data:  plms 
F = 27.7486, df1 = 27, df2 = 170, p-value < 2.2e-16
\end{Soutput}
\begin{Sinput}
> pooltest(zplm, znp, effect = T)
\end{Sinput}
\begin{Soutput}
	F statistic

data:  plms 
F = 5.7805, df1 = 18, df2 = 170, p-value = 1.219e-10
\end{Soutput}
\begin{Sinput}
> pooltest(zplm$within, znp)
\end{Sinput}
\begin{Soutput}
	F statistic

data:  plms 
F = 5.7805, df1 = 18, df2 = 170, p-value = 1.219e-10
\end{Soutput}
\begin{Sinput}
> z = plm(form, data = Grunfeld, effect = "time")
> znpt = pvcm(form, data = Grunfeld, effect = "time", model = "within")
> pooltest(z, znpt, effect = F)
\end{Sinput}
\begin{Soutput}
	F statistic

data:  plms 
F = 1.1204, df1 = 57, df2 = 140, p-value = 0.2928
\end{Soutput}
\end{Schunk}


\subsection{Tests for individual and time effects}



\subsubsection{Lagrange multiplier tests}

\texttt{plmtest} implements tests of individual or/and time effects  based on the results
of the pooling model. It's main argument is a
\texttt{plm} object (the result of a pooling model) or a
\texttt{plms} object.

Two additional arguments can be added to indicate the kind of test to
be computed. The argument \texttt{type} is whether :

\begin{itemize}
\item \texttt{bp} : \textsc{Breusch--Pagan} (1980), the default value,
\item \texttt{honda} : \textsc{Honda} (1985),
\item \texttt{kw} : \textsc{King} and \textsc{Wu} (1997).
\end{itemize}

The effects tested are indicated with the  \texttt{effect} argument :

\begin{itemize}
\item \texttt{individual} for individual effects  (the default value),
\item \texttt{time} for time effects,
\item \texttt{twoways} for individuals and time effects.
\end{itemize}

Some examples of the use of \texttt{plmtest} are shown below\footnote{See \textsc{Baltagi} (2001), p. 65.}:

\begin{Schunk}
\begin{Sinput}
> library(Ecdat)
> g <- plm(inv ~ value + capital, data = Grunfeld)
> plmtest(g)
\end{Sinput}
\begin{Soutput}
	Lagrange Multiplier Test - individual effects (Breush-Pagan)

data:  Grunfeld 
chi2 = 798.1615, df = 1, p-value < 2.2e-16
\end{Soutput}
\begin{Sinput}
> plmtest(g, effect = "time")
\end{Sinput}
\begin{Soutput}
	Lagrange Multiplier Test - time effects (Breush-Pagan)

data:  Grunfeld 
chi2 = 6.4539, df = 1, p-value = 0.01107
\end{Soutput}
\begin{Sinput}
> plmtest(g, type = "honda")
\end{Sinput}
\begin{Soutput}
	Lagrange Multiplier Test - individual effects (Honda)

data:  Grunfeld 
normal = 28.2518, p-value < 2.2e-16
\end{Soutput}
\begin{Sinput}
> plmtest(g, type = "ghm", effect = "twoways")
\end{Sinput}
\begin{Soutput}
	Lagrange Multiplier Test - two-ways effects (Gourierroux, Holly and
	Monfort)

data:  Grunfeld 
chi2 = 798.1615, df = 2, p-value < 2.2e-16
\end{Soutput}
\begin{Sinput}
> plmtest(g, type = "kw", effect = "twoways")
\end{Sinput}
\begin{Soutput}
	Lagrange Multiplier Test - two-ways effects (King and Wu)

data:  Grunfeld 
normal = 21.8322, df = 2, p-value < 2.2e-16
\end{Soutput}
\end{Schunk}

\subsubsection{F tests}

\texttt{pFtest} computes F tests of effects based on the comparison of
the  \texttt{within} and the \texttt{pooling} models. Its arguments
are whether a \texttt{plms} object or two \texttt{plm} objects
(the results of a  \texttt{pooling} and a \texttt{within} model).
Some examples of the use of \texttt{pFtest} are shown below\footnote{Voir \textsc{Baltagi} (2001),
  p. 65.}:

\begin{Schunk}
\begin{Sinput}
> library(Ecdat)
> gi <- plm(inv ~ value + capital, data = Grunfeld)
> gt <- plm(inv ~ value + capital, data = Grunfeld, effect = "time")
> gd <- plm(inv ~ value + capital, data = Grunfeld, effect = "twoways")
> pFtest(gi)
\end{Sinput}
\begin{Soutput}
	F test for effects

data:  gi 
F = 49.1766, df1 = 9, df2 = 188, p-value < 2.2e-16
\end{Soutput}
\begin{Sinput}
> pFtest(gi$within, gi$pooling)
\end{Sinput}
\begin{Soutput}
	F test for effects

data:  gi$within and gi$pooling 
F = 49.1766, df1 = 9, df2 = 188, p-value < 2.2e-16
\end{Soutput}
\begin{Sinput}
> pFtest(gt)
\end{Sinput}
\begin{Soutput}
	F test for effects

data:  gt 
F = 0.5229, df1 = 9, df2 = 188, p-value = 0.8569
\end{Soutput}
\begin{Sinput}
> pFtest(gd)
\end{Sinput}
\begin{Soutput}
	F test for effects

data:  gd 
F = 17.4031, df1 = 28, df2 = 169, p-value < 2.2e-16
\end{Soutput}
\end{Schunk}



\subsection{Hausman's test}

\texttt{phtest} computes the \textsc{Hausman}'s test which is based on
the  comparison of two models. It's main argument may be :

\begin{itemize}
\item a \texttt{plms} object. In this case, the two models used in the
  test are the \texttt{within} and the \texttt{random} models (the
  most usual case with panel data),
\item two \texttt{plm} objects.
\end{itemize}


Some examples of the use of \texttt{phtest} are shown below
\footnote{See \textsc{Baltagi} (2001), p. 71.}:


\begin{Schunk}
\begin{Sinput}
> g <- plm(inv ~ value + capital, data = Grunfeld)
> phtest(g)
\end{Sinput}
\begin{Soutput}
	Hausman Test

data:  g 
chi2 = 0.3638, df = 2, p-value = 0.8337
\end{Soutput}
\begin{Sinput}
> phtest(g$between, g$random)
\end{Sinput}
\begin{Soutput}
	Hausman Test

data:  g$between and g$random 
chi2 = -2.1314, df = 3, p-value = 1
\end{Soutput}
\end{Schunk}


\subsection{Robust covariance matrix estimation}
Robust estimators of the covariance matrix of coefficients are provided, mostly for use in Wald-type tests. \texttt{pvcovHC} estimates three ''flavours'' of White (1980, 1984)'s heteroskedasticity-consistent covariance matrix (known as the \emph{sandwich} estimator). Interestingly, in the context of panel data the most general version also proves consistent vs. serial correlation.

All types assume no correlation between errors of different groups while allowing for heteroskedasticity across groups, so that the full covariance matrix of errors is $  V=I_n \otimes \Omega_i;  i=1,..,n$. As for the \emph{intragroup} error covariance matrix of every single group of observations, \texttt{"white1"} allows for general heteroskedasticity but no serial correlation, i.e

\begin{equation}
 \label{eq:omegaW1}
 \Omega_i=
 \left[ \begin{array}{c c c c}
 \sigma_{i1}^2 & \dots & \dots & 0 \\
 0 & \sigma_{i2}^2 & & \vdots \\
 \vdots & & \ddots & 0 \\
 0 & & & \sigma_{iT}^2 \\
 \end{array} \right]
\end{equation}

while \texttt{"white2"} is \texttt{"white1"} restricted to a common variance inside every group, estimated as $\sigma_i^2=\sum_{t=1}^T{e_{it}^2}/T$, so that $\Omega_i=I_T \otimes \sigma_i^2$ (see Greene (2003), 13.7.1-2 and Wooldridge (2003), 10.7.2); \texttt{"arellano"} (see ibid. and the original ref. Arellano (1987)) allows a fully general structure w.r.t. heteroskedasticity and serial correlation:

\begin{equation}
 \label{eq:omegaArellano}
 \Omega_i=
 \left[ \begin{array}{c c c c c}
 \sigma_{i1}^2 & \sigma_{i1,i2}  & \dots & \dots & \sigma_{i1,iT} \\
 \sigma_{i2,i1} & \sigma_{i2}^2 & & & \vdots \\
 \vdots & & \ddots & & \vdots \\
 \vdots & & & \sigma_{iT-1}^2 & \sigma_{iT-1,iT} \\
 \sigma_{iT,i1} & \dots & \dots & \sigma_{iT,iT-1} & \sigma_{iT}^2 \\
 \end{array} \right]
\end{equation}

The latter is, as already observed, consistent w.r.t. timewise correlation of the errors, but on the converse, unlike the White 1 and 2 methods, it relies on large N asymptotics with small T. 

The errors may be weighted according to the schemes proposed by MacKinnon and White (1985) and Cribari-Neto (2004) to improve small-sample performance. 

Main use of \texttt{pvcovHC} is together with testing functions from \texttt{lmtest} and \texttt{car} packages. These typically allow passing the \texttt{vcov} parameter to be either a matrix or a function (see Zeileis 2004). If one is happy with the defaults, it is easiest to pass the function itself:

\begin{Schunk}
\begin{Sinput}
> library(lmtest)
> data(Airline)
> pdata.frame(Airline, "airline", "year")
> form <- log(cost) ~ log(output) + log(pf) + lf
> z <- plm(form, data = Airline, model = "within")
> coeftest(z, pvcovHC)
\end{Sinput}
\begin{Soutput}
t test of coefficients:

             Estimate Std. Error t value  Pr(>|t|)    
log(output)  0.919285   0.019105 48.1165 < 2.2e-16 ***
log(pf)      0.417492   0.013533 30.8507 < 2.2e-16 ***
lf          -1.070396   0.216620 -4.9413  4.11e-06 ***
---
Signif. codes:  0 ‘***’ 0.001 ‘**’ 0.01 ‘*’ 0.05 ‘.’ 0.1 ‘ ’ 1 
\end{Soutput}
\end{Schunk}

else one may do the covariance computation inside the call to \texttt{coeftest}, thus passing on a matrix:

\begin{Schunk}
\begin{Sinput}
> coeftest(z, pvcovHC(z, type = "white2", weights = "HC3"))
\end{Sinput}
\begin{Soutput}
t test of coefficients:

             Estimate Std. Error t value  Pr(>|t|)    
log(output)  0.919285   0.029021 31.6769 < 2.2e-16 ***
log(pf)      0.417492   0.014301 29.1928 < 2.2e-16 ***
lf          -1.070396   0.211686 -5.0565 2.605e-06 ***
---
Signif. codes:  0 ‘***’ 0.001 ‘**’ 0.01 ‘*’ 0.05 ‘.’ 0.1 ‘ ’ 1 
\end{Soutput}
\end{Schunk}

For some tests, e.g. for multiple model comparisons by \texttt{waldtest}, one should always provide a function\footnote{Joint zero-restriction testing still allows providing the \texttt{vcov} of the unrestricted model as a matrix, see the documentation of package \texttt{lmtest}}. In this case, optional parameters are provided as shown below (see also Zeileis, 2004, p.12):

\begin{Schunk}
\begin{Sinput}
> waldtest(z, update(z, . ~ . - log(pf) - lf), vcov = function(x) pvcovHC(x, 
+     type = "white2", weights = "HC3"))
\end{Sinput}
\begin{Soutput}
Wald test

Model 1: log(cost) ~ log(output) + log(pf) + lf
Model 2: log(cost) ~ log(output)
  Res.Df Df      F    Pr(>F)    
1     81                        
2     83 -2 429.46 < 2.2e-16 ***
---
Signif. codes:  0 ‘***’ 0.001 ‘**’ 0.01 ‘*’ 0.05 ‘.’ 0.1 ‘ ’ 1 
\end{Soutput}
\end{Schunk}

\texttt{linear.hypothesis} from package \texttt{car} may be used to test for linear restrictions:

\begin{Schunk}
\begin{Sinput}
> library(car)
> linear.hypothesis(zz, "2*log(pc)=log(emp)", vcov = pvcovHC)
\end{Sinput}
\begin{Soutput}
Linear hypothesis test

Hypothesis:
2 log(pc) - log(emp) = 0

Model 1: log(gsp) ~ log(pcap) + log(pc) + log(emp) + unemp
Model 2: restricted model

Note: Coefficient covariance matrix supplied.

  Res.Df  Df  Chisq Pr(>Chisq)    
1    812                          
2    813  -1 25.428  4.592e-07 ***
---
Signif. codes:  0 ‘***’ 0.001 ‘**’ 0.01 ‘*’ 0.05 ‘.’ 0.1 ‘ ’ 1 
\end{Soutput}
\end{Schunk}

\section{Bibiographie}

\setlength{\parindent}{0em}
\setlength{\parskip}{0.4cm}

  Amemiyia, T. (1971), The estimation of the variances in a
  variance--components model, \emph{International Economic Review}, 12,
  pp.1--13.

  Arellano M. (1987), Computing robust standard errors for within group estimators, 
\emph{Oxford bulletin of Economics and Statistics}, \bold{49}, 431--434.


  Arellano, M. (2003), Panel data econometrics, Oxford University Press.

  Arellano M. and S. Bond (1991), Some tests of specification for
  panel data : monte carlo evidence and an application to employment
  equations, \emph{Review of Economic Studies}, 58, pp.277--297.

  Balestra, P. and J. Varadharajan--Krishnakumar (1987), Full
  information estimations of a system of simultaneous equations with
  error components structure, \emph{Econometric Theory}, 3, pp.223--246.
  
  Baltagi, B.H. (1981), Simultaneous equations with error components,
  \emph{Journal of econometrics}, 17, pp.21--49.
  
  Baltagi, B.H. (2001) \emph{Econometric Analysis of Panel Data}. John
  Wiley and sons. ltd.

  Breusch, T.S. and A.R. Pagan (1980), The Lagrange multiplier test and
  its applications to model specification in econometrics, \emph{Review
    of Economic Studies}, 47, pp.239--253.

  Cribari-Neto F. (2004), Asymptotic inference under heteroskedasticity
of unknown form. \emph{Computational Statistics \& Data Analysis}
\bold{45}, 215--233.

  Gourieroux, C., A. Holly and A. Monfort (1982), Likelihood ratio test,
  Wald test, and Kuhn--Tucker test in linear models with inequality
  constraints on the regression parameters, \emph{Econometrica}, 50,
  pp.63--80.

  Greene W. H. (2003), \emph{Econometric Analysis}, 5th ed. Prentice Hall.

  Hausman, G. (1978), Specification tests in econometrics,
  \emph{Econometrica}, 46, pp.1251--1271.

  Hausman, J.A. and W.E. Taylor (1981), Panel data and unobservable
  individual effects, \emph{Econometrica}, 49, pp.1377--1398.
  
  Honda, Y. (1985), Testing the error components model with non--normal
  disturbances, \emph{Review of Economic Studies}, 52, pp.681--690.

  King, M.L. and P.X. Wu (1997), Locally optimal one--sided tests for
  multiparameter hypotheses, \emph{Econometric Reviews}, 33,
  pp.523--529.

  MacKinnon J. G., White H. (1985), Some heteroskedasticity-consistent
covariance matrix estimators with improved finite sample properties.
\emph{Journal of Econometrics} \bold{29}, 305--325.

  Nerlove, M. (1971), Further evidence on the estimation of dynamic
  economic relations from a time--series of cross--sections,
  \emph{Econometrica}, 39, pp.359--382.

  Swamy, P.A.V.B. (1970), Efficient inference in a random coefficient
  regression model, \emph{Econometrica}, 38, pp.311-323.

  
  Swamy, P.A.V.B. and S.S. Arora (1972), The exact finite sample
  properties of the estimators of coefficients in the error components
  regression models, \emph{Econometrica}, 40, pp.261--275.

  Wallace, T.D. and A. Hussain (1969), The use of error components
  models in combining cross section with time series data,
  \emph{Econometrica}, 37(1), pp.55--72.

  White H. (1980), \emph{Asymptotic Theory for Econometricians}, Ch. 6, Academic Press, Orlando (FL).

  White H. (1984), A heteroskedasticity-consistent covariance matrix and
a direct test for heteroskedasticity. \emph{Econometrica} \bold{48},
817--838.

  Wooldridge J. M. (2003), \emph{Econometric Analysis of Cross Section and Panel Data}, MIT Press

  Zeileis A. (2004), Econometric Computing with HC and HAC Covariance Matrix
Estimators. \emph{Journal of Statistical Software}, \bold{11}(10), 1--17.

URL \url{http://http://www.jstatsoft.org/v11/i10/}.

\end{document}


