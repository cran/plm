%\VignetteIndexEntry{Introduction a plm}
\documentclass[a4paper]{article}
\usepackage[francais]{babel}
\usepackage[T1]{fontenc}
\usepackage{ucs}
\usepackage[utf8]{inputenc}

\title{Introduction à plm}

\author{Yves Croissant}
\usepackage{/usr/share/R/share/texmf/Sweave}
\begin{document}

\maketitle


\section{Introduction}

L'objectif du package \texttt{plm} est de permettre d'estimer
simplement les modèles de panel. Le package \texttt{nlme}
(\textit{non--linear mixed effect models}) permet
d'estimer certains modèles de panel, mais pas de manière intuitive
pour un économètre. \texttt{plm} fournit des fonctions pour lire
des données de panel, pour estimer différents modèles et pour
réaliser des tests.

Ce package est chargé en utilisant :

\begin{Schunk}
\begin{Sinput}
> library(plm)
\end{Sinput}
\end{Schunk}


Ce document présente différents exemples d'utilisation de \texttt{plm}
en utilisant des données disponibles dans le package
\texttt{Ecdat}. 

\begin{Schunk}
\begin{Sinput}
> library(Ecdat)
\end{Sinput}
\end{Schunk}

Ces données sont utilisées dans \textsc{Baltagi} (2001).

\section{Lecture des données}

Avec \texttt{plm}, les données de panel sont stockées dans un objet de
type \texttt{pdata.frame}, qui est un \texttt{data.frame} auquel
plusieurs attributs ont été ajoutés de manière à stocker des
informations sur la structure des données utiles pour les
estimations. Un \texttt{pdata.frame} peut être créé à partir d'un
\texttt{data.frame} ordinaire (commande \texttt{pdata.frame}) ou
directement à partir d'un fichier texte (commande
\texttt{pread.table}). 


\subsection{Lire les données à partir d'un data.frame}



Nous illustrerons la commande
\texttt{pdata.frame} en utilisant les données \texttt{Produc} :


\begin{Schunk}
\begin{Sinput}
> data(Produc)
> pdata.frame(Produc, "state", "year", "pprod")
\end{Sinput}
\end{Schunk}

La commande \texttt{pdata.frame} attend 4 arguments :

\begin{itemize}
\item le nom du \texttt{data.frame},
\item \texttt{id} : la variable qui identifie les individus,
\item \texttt{time} : la variable qui identifie les périodes,
\item \texttt{name} : le nom sous lequel sera stocké le \texttt{pdata.frame}
\end{itemize}

Les observations doivent être classées par individu, puis par période. Le
troisième argument est optionnel, en son absence, une nouvelle série
appelée \texttt{time} est ajoutée aux données. Le quatrième argument
est également optionnel, en son absence, le \texttt{pdata.frame} est
stocké sous le même nom que le \texttt{data.frame}.
S'il n'y a pas d'index temporel, seul l'argument \texttt{id} peut être
renseigné, dans ce cas une variable appelée \texttt{time} est ajoutée
au \texttt{pdata.frame}.


\begin{Schunk}
\begin{Sinput}
> data(Hedonic)
> pdata.frame(Hedonic, "townid")
\end{Sinput}
\end{Schunk}

Enfin, il est possible, si le panel est cylindré, d'indiquer comme
deuxième argument non pas le nom de l'index individuel, mais le nombre
d'individus. Dans ce cas, deux séries supplémentaires \texttt{id} et
\texttt{time} sont créées :

\begin{Schunk}
\begin{Sinput}
> data(Wages)
> pdata.frame(Wages, 595)
\end{Sinput}
\end{Schunk}

Un résumé des données est obtenu en appliquant la méthode
\texttt{summmary} :

\begin{Schunk}
\begin{Sinput}
> summary(Hedonic)
\end{Sinput}
\begin{Soutput}
________________________________________________________________________________ 
___________________________________ Indexes ____________________________________
Individual index : townid
Time index       : time
________________________________________________________________________________ 
_______________________________ Panel Dimensions _______________________________
Unbalanced Panel
Number of Individuals        :  92
Number of Time Obserbations  :  from 1 to 30
Total Number of Observations :  506
________________________________________________________________________________ 
__________________________ Time/Individual Variation ___________________________
no time variation   :  zn indus rad tax ptratio 
________________________________________________________________________________ 
____________________________ Descriptive Statistics ____________________________
       mv              crim                zn             indus        chas    
 Min.   : 8.517   Min.   : 0.00632   Min.   :  0.00   Min.   : 0.46   no :471  
 1st Qu.: 9.742   1st Qu.: 0.08205   1st Qu.:  0.00   1st Qu.: 5.19   yes: 35  
 Median : 9.962   Median : 0.25651   Median :  0.00   Median : 9.69            
 Mean   : 9.942   Mean   : 3.61352   Mean   : 11.36   Mean   :11.14            
 3rd Qu.:10.127   3rd Qu.: 3.67708   3rd Qu.: 12.50   3rd Qu.:18.10            
 Max.   :10.820   Max.   :88.97620   Max.   :100.00   Max.   :27.74            
                                                                               
      nox              rm             age              dis        
 Min.   :14.82   Min.   :12.68   Min.   :  2.90   Min.   :0.1219  
 1st Qu.:20.16   1st Qu.:34.64   1st Qu.: 45.02   1st Qu.:0.7420  
 Median :28.94   Median :38.55   Median : 77.50   Median :1.1655  
 Mean   :32.11   Mean   :39.99   Mean   : 68.57   Mean   :1.1880  
 3rd Qu.:38.94   3rd Qu.:43.87   3rd Qu.: 94.07   3rd Qu.:1.6464  
 Max.   :75.86   Max.   :77.09   Max.   :100.00   Max.   :2.4954  
                                                                  
      rad             tax           ptratio          blacks       
 Min.   :0.000   Min.   :187.0   Min.   :12.60   Min.   :0.00032  
 1st Qu.:1.386   1st Qu.:279.0   1st Qu.:17.40   1st Qu.:0.37538  
 Median :1.609   Median :330.0   Median :19.05   Median :0.39144  
 Mean   :1.868   Mean   :408.2   Mean   :18.46   Mean   :0.35667  
 3rd Qu.:3.178   3rd Qu.:666.0   3rd Qu.:20.20   3rd Qu.:0.39623  
 Max.   :3.178   Max.   :711.0   Max.   :22.00   Max.   :0.39690  
                                                                  
     lstat             townid         time    
 Min.   :-4.0582   29     : 30   1      : 92  
 1st Qu.:-2.6659   84     : 23   2      : 75  
 Median :-2.1747   5      : 22   3      : 60  
 Mean   :-2.2342   83     : 19   4      : 50  
 3rd Qu.:-1.7744   41     : 18   5      : 39  
 Max.   :-0.9684   28     : 15   6      : 33  
                   (Other):379   (Other):157  
\end{Soutput}
\end{Schunk}

L'affichage se décompose en quatre sections :

\begin{itemize}
\item \texttt{indexes} indique le nom des index individuels et temporels,
\item \texttt{panel dimensions} donne des informations sur la
  dimension du  panel,
\item \texttt{Time/individual variation} indique les variables pour
lesquelles il n'y a pas de variation temporelle ou individuelle,
\item \texttt{Descriptive statistics} présente des statistiques
descriptives sur les différentes variables.
\end{itemize}

\subsection{Lire les données à partir d'un fichier texte}

La commande \texttt{pread.table} est proposée afin de lire des données
de panel directement à partir d'un fichier texte. La syntaxe est la
suivante :

\begin{verbatim}
pread.table("c:/mes documents/essai/donnees.txt",
         "firm","year","dataname",header=T,sep=";",dec=",")
\end{verbatim}

Les arguments de \texttt{pread.table} sont les suivants :

\begin{itemize}
\item le fichier texte qui contient les données,
\item \texttt{id} : la variable qui identifie les individus,
\item \texttt{time} : la variable qui identifie les périodes,
\item \texttt{name} : le nom du \texttt{pdata.frame} qui sera créé (si l'argument est nul, le nom du \texttt{pdata.frame}
  est le nom du fichier sans l'extension et sans le chemin d'accès),
\item d'autres arguments éventuels qui seront passés à la fonction
  \texttt{read.table} (ici, on précise que la première ligne du
  fichier contient le nom des variables, que le séparateur de champs
  est le point--virgule et que le séparateur décimal est la virgule).
\end{itemize}




\section{Estimer un modèle}

L'estimation d'un modèle de panel est obtenu avec la fonction
\texttt{plm}.


\subsection{Utilisation simple de plm}

Il y deux possibilités pour utiliser \texttt{plm} : la première
consiste à estimer une liste de modèles (comportement par défaut), la
seconde à estimer un modèle. Dans le premier cas, les modèles estimés
sont :

\begin{itemize}
\item le modèle à effets fixes (\texttt{within}),
\item les moindres carrés ordinaires (\texttt{pooling}),
\item le modèle estimé sur les moyennes individuelles ou temporelles (\texttt{between}),
\item le modèle à erreurs composées (\texttt{random}).
\end{itemize}

Il peut être pertinent d'estimer directement ces quatre modèles car
beaucoup de tests impliquent plusieurs d'entre eux. L'utilisation
la plus simple de \texttt{plm} consiste à indiquer une formule
décrivant le modèle à estimer et le \texttt{pdata.frame} qui contient
les données\footnote{L'exemple présenté ci--dessous est utilisé par  \textsc{Baltagi} (2001), pp. 25--28.} :

\begin{Schunk}
\begin{Sinput}
> zz <- plm(log(gsp) ~ log(pcap) + log(pc) + log(emp) + unemp, 
+     data = pprod)
\end{Sinput}
\end{Schunk}

Le résultat de l'estimation est stocké dans un objet de classe
\texttt{plms}. Celui--ci est une liste contenant les résultats des 4
modèles estimés. Chacun de ceux--ci est un objet de classe
\texttt{plm} et  peut ensuite être extrait :

\begin{Schunk}
\begin{Sinput}
> zzwith <- zz$within
\end{Sinput}
\end{Schunk}


On peut également préciser un modèle à estimer en fixant l'argument
\texttt{model} de \texttt{plm} à : \texttt{within}, \texttt{between},
\texttt{random} ou \texttt{pooling}.

\begin{Schunk}
\begin{Sinput}
> zzra <- plm(log(gsp) ~ log(pcap) + log(pc) + log(emp) + unemp, 
+     data = pprod, model = "random")
\end{Sinput}
\end{Schunk}

Les objets de classe \texttt{plm} et \texttt{plms} sont dotés d'une
méthode \texttt{print} qui ressemble à celle de la commande
\texttt{lm} :


\begin{Schunk}
\begin{Sinput}
> print(zzra)
\end{Sinput}
\begin{Soutput}
Model Formula: log(gsp) ~ log(pcap) + log(pc) + log(emp) + unemp

Coefficients:
(intercept)   log(pcap)     log(pc)    log(emp)       unemp 
  2.1354110   0.0044386   0.3105484   0.7296705  -0.0061725 
\end{Soutput}
\end{Schunk}


Ils disposent également d'une méthode \texttt{summary} :
\begin{itemize}
\item pour les objets de classe \texttt{plms}, les coefficients et les
  écarts--type des modèles à effets fixes et à erreurs composées sont affichés, ainsi que
  plusieurs tests (test d'Hausman, test de multiplicateur de Lagrange,
  test de F).
\item pour les objets de classe \texttt{plm}, le tableau des
  coefficients est affiché, ainsi que différentes statistiques.
\end{itemize}


\begin{Schunk}
\begin{Sinput}
> summary(zz)
\end{Sinput}
\begin{Soutput}
______________________________________________________________________ 
_________________________ Model Description __________________________
Oneway (individual) effect

Model Formula        : log(gsp) ~ log(pcap) + log(pc) + log(emp) + 
                           unemp
______________________________________________________________________ 
__________________________ Panel Dimensions __________________________
Balanced Panel
Number of Individuals        :  48
Number of Time Obserbations  :  17
Total Number of Observations :  816
______________________________________________________________________ 
____________________________ Coefficients ____________________________
                 within         wse      random    rse
(intercept)           .           .  2.13541100 0.1335
log(pcap)   -0.02614965  0.02900158  0.00443859 0.0234
log(pc)      0.29200693  0.02511967  0.31054843 0.0198
log(emp)     0.76815947  0.03009174  0.72967053 0.0249
unemp       -0.00529774  0.00098873 -0.00617247 0.0009
______________________________________________________________________ 
_______________________________ Tests ________________________________
Hausman Test                   : chi2(4) = 9.525416 (p.value=0.04922762)
F Test                         : F(47,764) = 75.8204 (p.value=0)
Lagrange Multiplier Test       : chi2(1) = 4134.961 (p.value=0)
______________________________________________________________________ 
\end{Soutput}
\begin{Sinput}
> summary(zzra)
\end{Sinput}
\begin{Soutput}
______________________________________________________________________ 
_________________________ Model Description __________________________
Oneway (individual) effect
Random Effect Model (Swamy-Arora's transformation)
Model Formula             : log(gsp) ~ log(pcap) + log(pc) + 
                                log(emp) + unemp
______________________________________________________________________ 
__________________________ Panel Dimensions __________________________
Balanced Panel
Number of Individuals        :  48
Number of Time Obserbations  :  17
Total Number of Observations :  816
______________________________________________________________________ 
______________________________ Effects _______________________________
                    var   std.dev  share
idiosyncratic 0.0014544 0.0381371 0.1754
individual    0.0068377 0.0826905 0.8246
theta   :  0.88884  
______________________________________________________________________ 
_____________________________ Residuals ______________________________
     Min.   1st Qu.    Median      Mean   3rd Qu.      Max. 
-1.07e-01 -2.46e-02 -2.37e-03 -9.93e-19  2.17e-02  2.00e-01 
______________________________________________________________________ 
____________________________ Coefficients ____________________________
               Estimate  Std. Error z-value  Pr(>|z|)    
(intercept)  2.13541100  0.13346149 16.0002 < 2.2e-16 ***
log(pcap)    0.00443859  0.02341732  0.1895    0.8497    
log(pc)      0.31054843  0.01980475 15.6805 < 2.2e-16 ***
log(emp)     0.72967053  0.02492022 29.2803 < 2.2e-16 ***
unemp       -0.00617247  0.00090728 -6.8033 1.023e-11 ***
---
Signif. codes:  0 ‘***’ 0.001 ‘**’ 0.01 ‘*’ 0.05 ‘.’ 0.1 ‘ ’ 1 
______________________________________________________________________ 
_________________________ Overall Statistics _________________________
Total Sum of Squares       : 29.209
Sum of Squares Residuals   : 1.1879
Rsq                        : 0.95933
F                          : 4782.77
P(F>0)                     : 8.76231e-08
______________________________________________________________________ 
\end{Soutput}
\end{Schunk}

Pour le modèle \texttt{random}, la commande \texttt{summary} donne des
précisions sur la variance des différents éléments du terme d'erreur.

Enfin, les objets issus de la commande \texttt{plm} supportent l'utilisation de
la commande \texttt{update}. Par exemple, pour enlever
\texttt{log(emp)} et \texttt{unemp} et ajouter \texttt{emp} à l'estimation précédente, on
utilisera la commande suivante :

\begin{Schunk}
\begin{Sinput}
> zzwithmod <- update(zzwith, . ~ . - unemp - log(emp) + emp)
> zzmod <- update(zz, . ~ . - unemp - log(emp) + emp)
> summary(zzwithmod)
\end{Sinput}
\begin{Soutput}
______________________________________________________________________ 
_________________________ Model Description __________________________
Oneway (individual) effect

Model Formula        : log(gsp) ~ log(pcap) + log(pc) + emp
______________________________________________________________________ 
__________________________ Panel Dimensions __________________________
Balanced Panel
Number of Individuals        :  48
Number of Time Obserbations  :  17
Total Number of Observations :  816
______________________________________________________________________ 
____________________________ Coefficients ____________________________
                within        wse     random       rse
(intercept)          .          . 7.1982e-01    0.1846
log(pcap)   1.7888e-01 4.0690e-02 3.4357e-01    0.0322
log(pc)     6.9975e-01 2.9154e-02 6.0369e-01    0.0256
emp         3.7909e-05 8.7824e-06 5.0924e-05 8.218e-06
______________________________________________________________________ 
_______________________________ Tests ________________________________
Hausman Test                   : chi2(3) = 80.35868 (p.value=0)
F Test                         : F(47,765) = 101.9109 (p.value=0)
Lagrange Multiplier Test       : chi2(1) = 4355.292 (p.value=0)
______________________________________________________________________ 
\end{Soutput}
\end{Schunk}


Les effets fixes peuvent être extraits facilement en utilisant la
commande \texttt{FE} qui prend comme argument soit un modèle estimé de
type \texttt{within}, soit un objet de classe \texttt{plms} :

\begin{Schunk}
\begin{Sinput}
> FE(zzmod)
\end{Sinput}
\begin{Soutput}
       ALABAMA        ARIZONA       ARKANSAS     CALIFORNIA       COLORADO 
     1.1717531      1.3062389      1.1877004      1.6191982      1.4582149 
   CONNECTICUT       DELAWARE        FLORIDA        GEORGIA          IDAHO 
     1.7060341      1.2035746      1.5564969      1.4460171      1.1002049 
      ILLINOIS        INDIANA           IOWA         KANSAS       KENTUCKY 
     1.5496106      1.3451714      1.2323038      1.1735476      1.3492604 
     LOUISIANA          MAINE       MARYLAND  MASSACHUSETTS       MICHIGAN 
     1.1652834      1.2659480      1.6011871      1.7384231      1.5290312 
     MINNESOTA    MISSISSIPPI       MISSOURI        MONTANA       NEBRASKA 
     1.3654287      1.1545345      1.4809262      0.7960951      1.0905033 
        NEVADA  NEW_HAMPSHIRE     NEW_JERSEY     NEW_MEXICO       NEW_YORK 
     1.0627992      1.4138235      1.7420589      1.0925399      1.6694387 
NORTH_CAROLINA   NORTH_DAKOTA           OHIO       OKLAHOMA         OREGON 
     1.5048751      0.7663694      1.4985974      1.2784660      1.3345094 
  PENNSYLVANIA   RHODE_ISLAND SOUTH_CAROLINA   SOUTH_DAKOTA       TENNESSE 
     1.4972243      1.5948140      1.2344011      0.8705826      1.3123010 
         TEXAS           UTAH        VERMONT       VIRGINIA     WASHINGTON 
     1.3230328      1.2464927      1.1804339      1.6175357      1.3492922 
 WEST_VIRGINIA      WISCONSIN        WYOMING 
     1.0129871      1.4860561      0.7842841 
\end{Soutput}
\end{Schunk}

\subsection{Options concernant le modèle à erreurs composées}

Le modèle à erreurs composées est estimé en réalisant une estimation
linéaire sur les données en quasi--différences. Le coefficient de
cette quasi--différence est obtenu à partir d'estimations préalables
des variances des différents éléments du terme d'erreur. Quatre
possibilités existent pour estimer ce coefficient, en fixant
l'argument \texttt{theta.method} à :

\begin{itemize}
\item \texttt{swar} : d'après \textsc{Swamy} et \textsc{Arora} (1972),
\item \texttt{walhus} : d'après \textsc{Wallace} et \textsc{Hussain} (1969),
\item \texttt{amemiya} : d'après \textsc{Amemiyia} (1971),
\item \texttt{nerlove} : d'après \textsc{Nerlove} (1971).
\end{itemize}

Par défaut, l'estimateur \texttt{swar} est utilisé. Pour, par exemple,
utiliser l'estimateur \texttt{amemiya}, on utilisera :

\begin{Schunk}
\begin{Sinput}
> zzra <- plm(log(gsp) ~ log(pcap) + log(pc) + log(emp) + unemp, 
+     data = pprod, model = "random", theta.method = "amemiya")
\end{Sinput}
\end{Schunk}


\subsection{Préciser les effets}

Par défaut, le modèle estimé par \texttt{plm} contient des effets
individuels. Il est également possible, via l'argument \texttt{effect}
d'introduire :


\begin{itemize}
\item des effets temporels \texttt{effect="time"},
\item des effets temporels et individuels \texttt{effect="twoways"}.
\end{itemize}

Par exemple, pour estimer un modèle à erreurs composées à effets
temporels et individuels pour les données \texttt{Grunfeld}, on
utilisera :

\begin{Schunk}
\begin{Sinput}
> data(Grunfeld)
> pdata.frame(Grunfeld, "firm", "year")
> z <- plm(inv ~ value + capital, data = Grunfeld, effect = "twoways", 
+     theta.method = "amemiya")
> summary(z$random)
\end{Sinput}
\begin{Soutput}
______________________________________________________________________ 
_________________________ Model Description __________________________
Twoways effects
Random Effect Model (Swamy-Arora's transformation)
Model Formula             : inv ~ value + capital
______________________________________________________________________ 
__________________________ Panel Dimensions __________________________
Balanced Panel
Number of Individuals        :  10
Number of Time Obserbations  :  20
Total Number of Observations :  200
______________________________________________________________________ 
______________________________ Effects _______________________________
                   var  std.dev  share
idiosyncratic 2675.426   51.725 0.2738
individual    7095.252   84.233 0.7262
time             0.000    0.000 0.0000
theta  : 0.86397 (id) 0 (time) 0 (total)
______________________________________________________________________ 
_____________________________ Residuals ______________________________
     Min.   1st Qu.    Median      Mean   3rd Qu.      Max. 
-1.77e+02 -1.98e+01  4.60e+00  8.77e-16  1.95e+01  2.53e+02 
______________________________________________________________________ 
____________________________ Coefficients ____________________________
              Estimate Std. Error z-value Pr(>|z|)    
(intercept) -57.865377  29.393359 -1.9687  0.04899 *  
value         0.109790   0.010528 10.4285  < 2e-16 ***
capital       0.308190   0.017171 17.9483  < 2e-16 ***
---
Signif. codes:  0 ‘***’ 0.001 ‘**’ 0.01 ‘*’ 0.05 ‘.’ 0.1 ‘ ’ 1 
______________________________________________________________________ 
_________________________ Overall Statistics _________________________
Total Sum of Squares       : 2376000
Sum of Squares Residuals   : 547910
Rsq                        : 0.7694
F                          : 328.647
P(F>0)                     : 0.0030381
______________________________________________________________________ 
\end{Soutput}
\end{Schunk}


Dans la section ``effects'' du résultat est désormais précisé
l'importance des trois éléments du terme d'erreur, ainsi que les trois
paramètres utilisés dans la transformation (associés aux moyennes
individuelle, temporelle et globale).

L'estimation du modèle à double effets (individuels et temporels)
n'est pour l'instant disponible que pour les panels cylindrés.


\subsection{Modèle de Hausman--Taylor}

Le modèle de \textsc{Hausman}--\textsc{Taylor} (1981) peut être estimé à l'aide de la commande
\texttt{plm} en fixant l'argument \texttt{model} à \texttt{"ht"} et en
précisant en second argument de \texttt{plm} une formule présentant les variables qui sont utilisées comme instruments
\footnote{L'exemple présenté ci--dessous est utilisé par
  \textsc{Baltagi} (2001), p. 130.} :


\begin{Schunk}
\begin{Sinput}
> data(Wages)
> pdata.frame(Wages, 595)
> form = lwage ~ wks + south + smsa + married + exp + I(exp^2) + 
+     bluecol + ind + union + sex + black + ed
> ht = plm(form, ~sex + black + bluecol + south + smsa + ind, data = Wages, 
+     model = "ht")
> summary(ht)
\end{Sinput}
\begin{Soutput}
______________________________________________________________________ 
_________________________ Model Description __________________________
Oneway (individual) effect
Hausman-Taylor Model
Model Formula             : lwage ~ wks + south + smsa + married + 
                                exp + I(exp^2) + bluecol + ind + 
                                union + sex + black + ed
Instrumental Variables    : ~sex + black + bluecol + south + 
                                smsa + ind
Time--Varying Variables    
    exogenous variables   :  bluecolyes,southyes,smsayes,ind 
    endogenous variables  :  wks,marriedyes,exp,I(exp^2),unionyes 
Time--Invariant Variables  
    exogenous variables   :  sexmale,blackyes 
    endogenous variables  :  ed 
______________________________________________________________________ 
__________________________ Panel Dimensions __________________________
Balanced Panel
Number of Individuals        :  595
Number of Time Obserbations  :  7
Total Number of Observations :  4165
______________________________________________________________________ 
______________________________ Effects _______________________________
                   var  std.dev  share
idiosyncratic 0.023044 0.151803 0.0253
individual    0.886993 0.941803 0.9747
theta   :  0.93919  
______________________________________________________________________ 
_____________________________ Residuals ______________________________
     Min.   1st Qu.    Median      Mean   3rd Qu.      Max. 
-1.92e+00 -7.07e-02  6.57e-03 -2.46e-17  7.97e-02  2.03e+00 
______________________________________________________________________ 
____________________________ Coefficients ____________________________
               Estimate  Std. Error z-value  Pr(>|z|)    
(intercept)  2.7818e+00  3.0768e-01  9.0411 < 2.2e-16 ***
wks          8.3740e-04  5.9981e-04  1.3961   0.16268    
southyes     7.4398e-03  3.1959e-02  0.2328   0.81592    
smsayes     -4.1833e-02  1.8960e-02 -2.2064   0.02736 *  
marriedyes  -2.9851e-02  1.8982e-02 -1.5726   0.11582    
exp          1.1313e-01  2.4713e-03 45.7795 < 2.2e-16 ***
I(exp^2)    -4.1886e-04  5.4605e-05 -7.6709 1.710e-14 ***
bluecolyes  -2.0705e-02  1.3783e-02 -1.5022   0.13304    
ind          1.3604e-02  1.5239e-02  0.8927   0.37202    
unionyes     3.2771e-02  1.4910e-02  2.1979   0.02796 *  
sexmale      1.3092e-01  1.2667e-01  1.0335   0.30135    
blackyes    -2.8575e-01  1.5572e-01 -1.8350   0.06651 .  
ed           1.3794e-01  2.1251e-02  6.4912 8.518e-11 ***
---
Signif. codes:  0 ‘***’ 0.001 ‘**’ 0.01 ‘*’ 0.05 ‘.’ 0.1 ‘ ’ 1 
______________________________________________________________________ 
_________________________ Overall Statistics _________________________
Total Sum of Squares       : 243.04
Sum of Squares Residuals   : 95.947
Rsq                        : 0.60522
F                          : 530.318
P(F>0)                     : 2.88658e-15
______________________________________________________________________ 
\end{Soutput}
\end{Schunk}

\subsection{Variables instrumentales}

Un ou l'ensemble des modèles de panel peut être estimé avec des
variables instrumentales, en précisant la liste des variables
utilisées comme instruments. Les instruments peuvent être précisés de
deux manières différentes :

\begin{itemize}
\item en spécifiant la liste totale des instruments (à l'aide de
  l'argument \texttt{instrument} de \texttt{plm}),
\item en spécifiant, d'une part la liste des instruments extérieurs
  au modèle (argument \texttt{instrument}) et d'autre part la liste des variables du modèles qui sont
  considérées comme corrélées avec le terme d'erreur (argument \texttt{endog}).
\end{itemize}

La méthode utilisée peut être spécifiée à l'aide de l'argument
\texttt{inst.method} :

\begin{itemize}
\item \texttt{bvk}, d'après \textsc{Balestra} et \textsc{Varadharajan--Krishnakumar} (1987), la valeur par défaut,
\item \texttt{baltagi}, d'après \textsc{Baltagi} (1981).
\end{itemize}

Nous illustrons l'estimation d'un modèle de panel avec variables
instrumentales avec les données \texttt{Crime}\footnote{Voir
  \textsc{Baltagi} (2001), pp.119--120.}. La même estimation est
réalisée en utilisant la première technique (\texttt{cr1}) et la
deuxième (\texttt{cr2}). Les variables \texttt{prbarr} et \texttt{polpc}
sont considérées comme endogènes et il y a deux instruments extérieurs
au modèle,
\texttt{taxpc} et \texttt{mix} :

\begin{Schunk}
\begin{Sinput}
> data(Crime)
> pdata.frame(Crime, "county", "year")
> form = log(crmrte) ~ log(prbarr) + log(polpc) + log(prbconv) + 
+     log(prbpris) + log(avgsen) + log(density) + log(wcon) + log(wtuc) + 
+     log(wtrd) + log(wfir) + log(wser) + log(wmfg) + log(wfed) + 
+     log(wsta) + log(wloc) + log(pctymle) + log(pctmin) + region + 
+     smsa + year
> inst = ~log(prbconv) + log(prbpris) + log(avgsen) + log(density) + 
+     log(wcon) + log(wtuc) + log(wtrd) + log(wfir) + log(wser) + 
+     log(wmfg) + log(wfed) + log(wsta) + log(wloc) + log(pctymle) + 
+     log(pctmin) + region + smsa + log(taxpc) + log(mix) + year
> inst2 = ~log(taxpc) + log(mix)
> endog = ~log(prbarr) + log(polpc)
> cr = plm(form, data = Crime)
> cr1 = plm(form, inst, data = Crime)
> cr2 = plm(form, inst2, endog, data = Crime)
> summary(cr2$random)
\end{Sinput}
\begin{Soutput}
______________________________________________________________________ 
_________________________ Model Description __________________________
Oneway (individual) effect
Random Effect Model (Swamy-Arora's transformation)
Instrumental variable estimation (Balestra-Varadharajan-Krishnakumar's transformation)
Model Formula             : log(crmrte) ~ log(prbarr) + log(polpc) + 
                                log(prbconv) + log(prbpris) + 
                                log(avgsen) + log(density) + 
                                log(wcon) + log(wtuc) + log(wtrd) + 
                                log(wfir) + log(wser) + log(wmfg) + 
                                log(wfed) + log(wsta) + log(wloc) + 
                                log(pctymle) + log(pctmin) + 
                                region + smsa + year
Endogenous Variables    : ~log(prbarr) + log(polpc)
Instrumental Variables    : ~log(taxpc) + log(mix)
______________________________________________________________________ 
__________________________ Panel Dimensions __________________________
Balanced Panel
Number of Individuals        :  90
Number of Time Obserbations  :  7
Total Number of Observations :  630
______________________________________________________________________ 
______________________________ Effects _______________________________
                   var  std.dev share
idiosyncratic 0.022269 0.149228 0.326
individual    0.046036 0.214561 0.674
theta   :  0.74576  
______________________________________________________________________ 
_____________________________ Residuals ______________________________
     Min.   1st Qu.    Median      Mean   3rd Qu.      Max. 
-5.02e+00 -4.76e-01  2.73e-02  7.11e-16  5.26e-01  3.19e+00 
______________________________________________________________________ 
____________________________ Coefficients ____________________________
                Estimate Std. Error z-value  Pr(>|z|)    
(intercept)   -0.4538241  1.7029840 -0.2665  0.789864    
log(prbarr)   -0.4141200  0.2210540 -1.8734  0.061015 .  
log(polpc)     0.5049285  0.2277811  2.2167  0.026642 *  
log(prbconv)  -0.3432383  0.1324679 -2.5911  0.009567 ** 
log(prbpris)  -0.1900437  0.0733420 -2.5912  0.009564 ** 
log(avgsen)   -0.0064374  0.0289406 -0.2224  0.823977    
log(density)   0.4343519  0.0711528  6.1045 1.031e-09 ***
log(wcon)     -0.0042963  0.0414225 -0.1037  0.917392    
log(wtuc)      0.0444572  0.0215449  2.0635  0.039068 *  
log(wtrd)     -0.0085626  0.0419822 -0.2040  0.838387    
log(wfir)     -0.0040302  0.0294565 -0.1368  0.891175    
log(wser)      0.0105604  0.0215822  0.4893  0.624620    
log(wmfg)     -0.2017917  0.0839423 -2.4039  0.016220 *  
log(wfed)     -0.2134634  0.2151074 -0.9924  0.321023    
log(wsta)     -0.0601083  0.1203146 -0.4996  0.617362    
log(wloc)      0.1835137  0.1396721  1.3139  0.188884    
log(pctymle)  -0.1458448  0.2268137 -0.6430  0.520214    
log(pctmin)    0.1948760  0.0459409  4.2419 2.217e-05 ***
regionwest    -0.2281780  0.1010317 -2.2585  0.023916 *  
regioncentral -0.1987675  0.0607510 -3.2718  0.001068 ** 
smsayes       -0.2595423  0.1499780 -1.7305  0.083535 .  
year82         0.0132140  0.0299923  0.4406  0.659518    
year83        -0.0847676  0.0320008 -2.6489  0.008075 ** 
year84        -0.1062004  0.0387893 -2.7379  0.006184 ** 
year85        -0.0977398  0.0511685 -1.9102  0.056113 .  
year86        -0.0719390  0.0605821 -1.1875  0.235045    
year87        -0.0396520  0.0758537 -0.5227  0.601153    
---
Signif. codes:  0 ‘***’ 0.001 ‘**’ 0.01 ‘*’ 0.05 ‘.’ 0.1 ‘ ’ 1 
______________________________________________________________________ 
_________________________ Overall Statistics _________________________
Total Sum of Squares       : 1354.7
Sum of Squares Residuals   : 557.64
Rsq                        : 0.58836
F                          : 33.1494
P(F>0)                     : 7.77156e-16
______________________________________________________________________ 
\end{Soutput}
\end{Schunk}


\subsection{Modèle à coefficients individuels}

Si le nombre d'observations temporelles est suffisant, le modèle peut
également être spécifié de manière à ce que les différents
coefficients soient a priori différents d'un individu à un autre. Dans
ce cas, la même équation est estimée pour chaque individu. Cette
estimation peut être réalisée avec \texttt{plm} de deux manières
différentes :

\begin{itemize}
\item en utilisant la commande \texttt{nopool},
\item en utilisant la commande \texttt{plm} avec l'argument
  \texttt{np} fixé à \texttt{TRUE}.
\end{itemize}


Avec les données \texttt{Grunfeld}, on obtient le résultat suivant :


\begin{Schunk}
\begin{Sinput}
> z <- nopool(inv ~ value + capital, data = Grunfeld)
> print(z)
\end{Sinput}
\begin{Soutput}
    (intercept)       value     capital
1  -149.7824533 0.119280833 0.371444807
2   -49.1983219 0.174856015 0.389641889
3    -9.9563065 0.026551189 0.151693870
4    -6.1899605 0.077947821 0.315718185
5    22.7071160 0.162377704 0.003101737
6    -8.6855434 0.131454842 0.085374274
7    -4.4995344 0.087527198 0.123781407
8    -0.5093902 0.052894126 0.092406492
9    -7.7228371 0.075387943 0.082103558
10    0.1615186 0.004573432 0.437369190
\end{Soutput}
\begin{Sinput}
> summary(z)
\end{Sinput}
\begin{Soutput}
  (intercept)           value             capital        
 Min.   :-149.782   Min.   :0.004573   Min.   :0.003102  
 1st Qu.:  -9.639   1st Qu.:0.058518   1st Qu.:0.087132  
 Median :  -6.956   Median :0.082738   Median :0.137738  
 Mean   : -21.368   Mean   :0.091285   Mean   :0.205264  
 3rd Qu.:  -1.507   3rd Qu.:0.128411   3rd Qu.:0.357513  
 Max.   :  22.707   Max.   :0.174856   Max.   :0.437369  
\end{Soutput}
\end{Schunk}


Le résultat est un objet de classe \texttt{nopool}. L'impression de
cet objet présente les coefficients estimés pour chaque individu. La
méthode \texttt{summary} présente des statistiques descriptives pour
ces coefficients.

Dans la commande ci--dessous, l'estimation est réalisée à partir de la
commande \texttt{plm} :


\begin{Schunk}
\begin{Sinput}
> z <- plm(inv ~ value + capital, data = Grunfeld, np = TRUE)
> znp <- z$nopool
\end{Sinput}
\end{Schunk}

Le résultat de l'estimation du modèle \texttt{nopool} est alors un
élément supplémentaire  de l'objet renvoyé par \texttt{plm}.



\subsection{Panel non--cylindré}

\texttt{plm} propose un support limité pour les panels non--cylindrés.
Nous présentons ci--dessous les résultats d'une estimation réalisée
sur les données \texttt{Hedonic}\footnote{Voir \textsc{Baltagi}
  (2001), p. 174.}.



\begin{Schunk}
\begin{Sinput}
> form = mv ~ crim + zn + indus + chas + nox + rm + age + dis + 
+     rad + tax + ptratio + blacks + lstat
> ba = plm(form, data = Hedonic)
> summary(ba$random)
\end{Sinput}
\begin{Soutput}
______________________________________________________________________ 
_________________________ Model Description __________________________
Oneway (individual) effect
Random Effect Model (Swamy-Arora's transformation)
Model Formula             : mv ~ crim + zn + indus + chas + nox + 
                                rm + age + dis + rad + tax + 
                                ptratio + blacks + lstat
______________________________________________________________________ 
__________________________ Panel Dimensions __________________________
Unbalanced Panel
Number of Individuals        :  92
Number of Time Obserbations  :  from 1 to 30
Total Number of Observations :  506
______________________________________________________________________ 
______________________________ Effects _______________________________
                   var  std.dev share
idiosyncratic 0.016965 0.130249 0.502
individual    0.016832 0.129738 0.498
theta  : 
   Min. 1st Qu.  Median    Mean 3rd Qu.    Max. 
 0.2915  0.5904  0.6655  0.6499  0.7447  0.8197 
______________________________________________________________________ 
_____________________________ Residuals ______________________________
     Min.   1st Qu.    Median      Mean   3rd Qu.      Max. 
-0.641000 -0.066100 -0.000519 -0.001990  0.069800  0.527000 
______________________________________________________________________ 
____________________________ Coefficients ____________________________
               Estimate  Std. Error  z-value  Pr(>|z|)    
(intercept)  9.6778e+00  2.0714e-01  46.7207 < 2.2e-16 ***
crim        -7.2338e-03  1.0346e-03  -6.9921 2.707e-12 ***
zn           3.9575e-05  6.8778e-04   0.0575 0.9541153    
indus        2.0794e-03  4.3403e-03   0.4791 0.6318706    
chasyes     -1.0591e-02  2.8960e-02  -0.3657 0.7145720    
nox         -5.8630e-03  1.2455e-03  -4.7074 2.509e-06 ***
rm           9.1773e-03  1.1792e-03   7.7828 7.105e-15 ***
age         -9.2715e-04  4.6468e-04  -1.9952 0.0460159 *  
dis         -1.3288e-01  4.5683e-02  -2.9088 0.0036279 ** 
rad          9.6863e-02  2.8350e-02   3.4168 0.0006337 ***
tax         -3.7472e-04  1.8902e-04  -1.9824 0.0474298 *  
ptratio     -2.9723e-02  9.7538e-03  -3.0473 0.0023089 ** 
blacks       5.7506e-01  1.0103e-01   5.6920 1.256e-08 ***
lstat       -2.8514e-01  2.3855e-02 -11.9533 < 2.2e-16 ***
---
Signif. codes:  0 ‘***’ 0.001 ‘**’ 0.01 ‘*’ 0.05 ‘.’ 0.1 ‘ ’ 1 
______________________________________________________________________ 
_________________________ Overall Statistics _________________________
Total Sum of Squares       : 893.08
Sum of Squares Residuals   : 8.6843
Rsq                        : 0.99028
F                          : 3854.18
P(F>0)                     : 0
______________________________________________________________________ 
\end{Soutput}
\end{Schunk}
\section{Tests}


\subsection{Tests de coefficients identiques}

On teste ici l'hypothèse que les mêmes coefficients associés à chaque
variable sont les mêmes pour tous les individus. Il s'agit d'un test
de F classique, comparant les résultats d'un modèle estimé globalement
pour l'ensemble des individus et d'un modèle obtenu en estimant une
équation pour chaque individu. La commande \texttt{pooltest} permet
de réaliser ce test. Cette commande prend pour principal  argument un
objet 
de classe \texttt{plms}. Si le modèle a été estimé par \texttt{plm}
avec l'argument \texttt{np=F}, il faut spécifier un deuxième argument
de classe \texttt{nopool} (dans le cas inverse, le modèle
\texttt{nopool} est un élément de l'objet \texttt{plms}).

Un troisième argument \texttt{effect} doit être fixé à vrai ou faux
(valeur par défaut), suivant que sous H$_\mtext{o$}, les constantes
individuelles sont supposées être différentes ou
identiques\footnote{Les exemples ci--dessous sont présentés par
  \textsc{Baltagi} (2001), pp. 57--58.}:

\begin{Schunk}
\begin{Sinput}
> form = inv ~ value + capital
> znp = nopool(form, data = Grunfeld)
> zplm = plm(form, data = Grunfeld)
> pooltest(zplm, znp)
\end{Sinput}
\begin{Soutput}
	F statistic

data:  zplm 
F = 27.7486, df1 = 27, df2 = 170, p-value < 2.2e-16
\end{Soutput}
\begin{Sinput}
> pooltest(zplm, znp, effect = T)
\end{Sinput}
\begin{Soutput}
	F statistic

data:  zplm 
F = 5.7805, df1 = 18, df2 = 170, p-value = 1.219e-10
\end{Soutput}
\begin{Sinput}
> z = plm(form, data = Grunfeld, effect = "time", np = TRUE)
> pooltest(z, effect = F)
\end{Sinput}
\begin{Soutput}
	F statistic

data:  z 
F = 1.1204, df1 = 57, df2 = 140, p-value = 0.2928
\end{Soutput}
\end{Schunk}


\subsection{Tests sur les effets}



\subsubsection{Tests de multiplicateur de Lagrange}

Ces tests de présence d'effets individuels et/ou temporels sont basés sur
les résultats du modèle de pooling. La commande \texttt{plmtest}
permet de calculer ces tests, en indiquant en argument soit un objet
de classe \texttt{plms}, soit un objet de classe \texttt{plm} (en
l'occurence le résultat d'un modèle \texttt{pooling}). 

Deux arguments supplémentaires permettent de préciser le test à
calculer. \texttt{type} peut être fixé à :

\begin{itemize}
\item \texttt{bp}, test de \textsc{Breusch--Pagan} (1980), la valeur par défaut,
\item \texttt{honda}, d'après \textsc{Honda} (1985),
\item \texttt{kw}, d'après \textsc{King} et \textsc{Wu} (1997).
\end{itemize}

Le type d'effet testé est précisé grâce à l'argument \texttt{effect}
fixé à :

\begin{itemize}
\item \texttt{individual} pour des effets individuels (la valeur par défaut),
\item \texttt{time} pour des effets temporels,
\item \texttt{twoways} pour des effets individuels et temporels.
\end{itemize}

Différents exemple d'utilisation de tests de multiplicateur de Lagrange
sont présentés ci--dessous\footnote{Voir \textsc{Baltagi} (2001), p. 65.}.

\begin{Schunk}
\begin{Sinput}
> library(Ecdat)
> g <- plm(inv ~ value + capital, data = Grunfeld)
> plmtest(g)
\end{Sinput}
\begin{Soutput}
	Lagrange Multiplier Test - individual effects (Breush-Pagan)

data:  g 
chi2 = 798.1615, df = 1, p-value < 2.2e-16
\end{Soutput}
\begin{Sinput}
> plmtest(g, effect = "time")
\end{Sinput}
\begin{Soutput}
	Lagrange Multiplier Test - time effects (Breush-Pagan)

data:  g 
chi2 = 6.4539, df = 1, p-value = 0.01107
\end{Soutput}
\begin{Sinput}
> plmtest(g, type = "honda")
\end{Sinput}
\begin{Soutput}
	Lagrange Multiplier Test - individual effects (Honda)

data:  g 
normal = 28.2518, p-value < 2.2e-16
\end{Soutput}
\begin{Sinput}
> plmtest(g, type = "ghm", effect = "twoways")
\end{Sinput}
\begin{Soutput}
	Lagrange Multiplier Test - two-ways effects (Gourierroux, Holly and
	Monfort)

data:  g 
chi2 = 798.1615, df = 2, p-value < 2.2e-16
\end{Soutput}
\begin{Sinput}
> plmtest(g, type = "kw", effect = "twoways")
\end{Sinput}
\begin{Soutput}
	Lagrange Multiplier Test - two-ways effects (King and Wu)

data:  g 
normal = 21.8322, df = 2, p-value < 2.2e-16
\end{Soutput}
\end{Schunk}



\subsubsection{Tests de F}

La présence d'effets peut également être testée à partir de la
comparaison des  résultats des modèles \texttt{within} et
\texttt{pooling}, sous la forme d'un test de F. La commande
\texttt{pFtest} permet de calculer ce type de test, en indiquant comme
argument soit un objet de classe \texttt{plms}, soit deux objets de
classe \texttt{plm}, contenant respectivement les résultats d'un
modèle \texttt{within} et d'un modèle \texttt{pooling}.
Différents exemples d'utilisation de \texttt{pFtest}
sont présentés ci--dessous\footnote{Voir \textsc{Baltagi} (2001),
  p. 65.}:

\begin{Schunk}
\begin{Sinput}
> library(Ecdat)
> gi <- plm(inv ~ value + capital, data = Grunfeld)
> gt <- plm(inv ~ value + capital, data = Grunfeld, effect = "time")
> gd <- plm(inv ~ value + capital, data = Grunfeld, effect = "twoways")
> pFtest(gi)
\end{Sinput}
\begin{Soutput}
	F test for effects

data:  gi 
F = 49.1766, df1 = 9, df2 = 188, p-value < 2.2e-16
\end{Soutput}
\begin{Sinput}
> pFtest(gi$within, gi$pooling)
\end{Sinput}
\begin{Soutput}
	F test for effects

data:  gi$within and gi$pooling 
F = 49.1766, df1 = 9, df2 = 188, p-value < 2.2e-16
\end{Soutput}
\begin{Sinput}
> pFtest(gt)
\end{Sinput}
\begin{Soutput}
	F test for effects

data:  gt 
F = 0.5229, df1 = 9, df2 = 188, p-value = 0.8569
\end{Soutput}
\begin{Sinput}
> pFtest(gd)
\end{Sinput}
\begin{Soutput}
	F test for effects

data:  gd 
F = 17.4031, df1 = 28, df2 = 169, p-value < 2.2e-16
\end{Soutput}
\end{Schunk}



\subsection{Test d'Hausman}

Le test d'\textsc{Hausman} (1978) est un test basé sur la comparaison de deux
modèles :

\begin{itemize}
\item sous l'hypothèse nulle, les deux modèles sont convergents et
  l'un des deux est plus efficace que l'autre,
\item sous l'hypothèse alternative, seul le premier des deux modèles
  est convergent.
\end{itemize}


La commande \texttt{phtest} permet de calculer le test d'Hausman. Elle
peut prendre comme argument un objet de classe \texttt{plms}, dans ce
cas les deux modèles concernés sont les modèles \texttt{within} et
\texttt{random} (usage le plus courant en économétrie de panel). Elle
peut également prendre pour argument deux objets de classe \texttt{plm}.
Différents exemples d'utilisation du test d'\textsc{Hausman}
sont présentés ci--dessous\footnote{Voir \textsc{Baltagi} (2001), p. 71.}.


\begin{Schunk}
\begin{Sinput}
> g <- plm(inv ~ value + capital, data = Grunfeld)
> phtest(g)
\end{Sinput}
\begin{Soutput}
	Hausman Test

data:  g 
chi2 = 2.3304, df = 2, p-value = 0.3119
\end{Soutput}
\begin{Sinput}
> phtest(g$between, g$random)
\end{Sinput}
\begin{Soutput}
	Hausman Test

data:  g$between and g$random 
chi2 = 2.1314, df = 3, p-value = 0.5456
\end{Soutput}
\end{Schunk}


\section{Bibiographie}

\setlength{\parindent}{0em}
\setlength{\parskip}{0.4cm}

  Amemiyia, T. (1971), The estimation of the variances in a
  variance--components model, \emph{International Economic Review}, 12,
  pp.1--13.

  Balestra, P. et J. Varadharajan--Krishnakumar (1987), Full
  information estimations of a system of simultaneous equations with
  error components structure, \emph{Econometric Theory}, 3, pp.223--246.
  
  Baltagi, B.H. (1981), Simultaneous equations with error components,
  \emph{Journal of econometrics}, 17, pp.21--49.
  
  Baltagi, B.H. (2001) \emph{Econometric Analysis of Panel Data}. John
  Wiley and sons. ltd.

  Breusch, T.S. et A.R. Pagan (1980), The Lagrange multiplier test and
  its applications to model specification in econometrics, \emph{Review
    of Economic Studies}, 47, pp.239--253.

  Gourieroux, C., A. Holly et A. Monfort (1982), Likelihood ratio test,
  Wald test, and Kuhn--Tucker test in linear models with inequality
  constraints on the regression parameters, \emph{Econometrica}, 50,
  pp.63--80.

  Hausman, G. (1978), Specification tests in econometrics,
  \emph{Econometrica}, 46, pp.1251--1271.

  Hausman, J.A. et W.E. Taylor (1981), Panel data and unobservable
  individual effects, \emph{Econometrica}, 49, pp.1377--1398.
  
  Honda, Y. (1985), Testing the error components model with non--normal
  disturbances, \emph{Review of Economic Studies}, 52, pp.681--690.

  King, M.L. et P.X. Wu (1997), Locally optimal one--sided tests for
  multiparameter hypotheses, \emph{Econometric Reviews}, 33,
  pp.523--529.

  Nerlove, M. (1971), Further evidence on the estimation of dynamic
  economic relations from a time--series of cross--sections,
  \emph{Econometrica}, 39, pp.359--382.

  Swamy, P.A.V.B. et S.S. Arora (1972), The exact finite sample
  properties of the estimators of coefficients in the error components
  regression models, \emph{Econometrica}, 40, pp.261--275.

  Wallace, T.D. et A. Hussain (1969), The use of error components
  models in combining cross section with time series data,
  \emph{Econometrica}, 37(1), pp.55--72.

\end{document}


